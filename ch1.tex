\section{Sets and Functions}
\subsection{Exercises 1.2}
\question{State precisely what it means when two sets A and B are not equal.}

\begin{solution}
  $A \neq B \implies$ either \\
  1) $A \subset B$ and $B \not\subset A$ or\\
  2) $B \subset A$ and $A \not\subset B$\\
\end{solution}

\setcounter{question}{0}
\subsection{Exercises 2.2}

\question{What is $\mathcal{P}(\{1,2,3\})$}
\begin{solution}
  $\{\emptyset, \{1\}, \{2\}, \{3\}, \{1,2\}, \{1,3\}, \{2,3\}, \{1,2,3\}\}$
\end{solution}

\question{How many elements does a power set of n elements have?}
\begin{solution}
  Observe that each element in the power set either contains a given element from the original set or it does not, hence there are 2 possibilities for each of the n elements. Therefore, a power set of n elements has $2^n$ elements.\\\textit{A formal proof likely using induction is left as an exercise.}
\end{solution}

\question{Are the following statements true?}
\begin{parts}
\part{$x\in X \iff \{x\}\in\mathcal{P}(X)$}}
\begin{solution}
  True.
\end{solution}
\part{$\{x\}\in\mathcal{P}(X) \iff \{x\}\subset X$}
\begin{solution}
  True. Consider $\{x\} \in \mathcal{P}(X)$. By definition $x \in X \implies \{x\} \subset X$.\\
  Also $\{x\} \subset X \implies x \in X \implies \{x\} \in \mathcal{P}(X)$ by definition.
\end{solution}
\part{$\{x\}\subset\mathcal{P}(X) \iff x\subset X$}
\begin{solution}
  False. $x\not\subset X$ by definition an element cannot be a subset.
\end{solution}
\end{parts}
