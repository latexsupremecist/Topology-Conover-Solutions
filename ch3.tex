\section{Some Familiar Topological Spaces and Basic Topological Concepts}

\subsection{Exercises 1.2}


\question{Let $A = (0,1) \cup (1,3)$. For the given $x \in A$, give a value of $r > 0$ such that $(x-r, x+r) \subseteq A$.}

\begin{parts}
 
 \part{$x = \frac{3}{4}$}
 \part{$x = 2$}
 \part{$x = \frac{9}{8}$}
 
\begin{solution}
 $\frac{1}{16}$
\end{solution}

\end{parts}

\question{Prove that A = $(0,1) \cup (1,3)$ is an open subset of $\mathbb{R}$.}

\begin{solution}
\\Case 1: $x \in (0,1)$ \\
Let $d = \text{min}\{1-x,x\}$. $(x-\frac{d}{2},x+\frac{d}{2}) \subseteq A$ is an open set. \\
Case 2: $x \in (1,3)$ \\
Let $d = \text{min}\{3-x,x-1\}$. The proof proceeds similarly as Case 1. \\
As $\forall x \in A \exists$ an open interval $\subseteq A$ which contains $x$, $A$ is an open set by definition.
\end{solution}

\question{Prove than an ordinary open interval is an open subset of $\mathbb{R}$ but that an open set need \textit{not} be an open interval.}

\begin{solution}
 \\All open intervals are open subsets: \\
 Replace 0 with $x-r$ and 1 with $x+r$ as in Case 1 from the question above. \\
 Open sets need not be an open interval: \\
 The empty set is an open set, but no open interval is empty, as it contains $x$ (open intervals are in the form $(x-r,x+r)$).
\end{solution}

\question{State precisely what it means when a subset $A$ of $\mathbb{R}$ is \textit{not} open.}

\begin{solution}
 $\exists x \in A$ such that $(x-r,x+r)-A \neq \emptyset \forall r > 0$
\end{solution}

\question{Prove that the following subsets of $\mathbb{R}$ are not open.}

\begin{parts}
 
 \part{The set of rational numbers}
 
 
\begin{solution}
 \\An open interval is nonempty. As it is an open set, it cannot contain a single point only (next question) so it must contain at least 2 points. WLOG, let the 2 points be $p < q$. Using limits, $\exists n \in \mathbb{N}$ such that $r = p + \frac{\sqrt{2}}{n} < q$. Since all open intervals contain an irrational $r$, no open intervals can be a subset of the rationals, hence it is not open.
\end{solution}

 \part{A set consisting of a single point}
 
 
\begin{solution}
 \\$\forall$ open intervals $(x-r,x+r)$, it contains the distinct points $x$ and $x+\frac{r}{2}$. Therefore, $\{x\}$ cannot be open.
\end{solution}

 \part{An interval of the form $[a,b)$, where $a < b$}
 
 
\begin{solution}
\\ $\forall$ open intervals $(a-r,a+r)$, $b=a-\frac{r}{2}$ is a point in that interval which is outside of $[a,b)$.
\end{solution}

 \part{The set $A = \{x\in\mathbb{R}:x\neq\frac{1}{n},\text{ for }n\in\mathbb{Z}^+\}$}
 
 
\begin{solution}
\\ $\frac{1}{n}$ tends to 0. Therefore $\forall r > 0 \exists n \in \mathbb{N}$ such that $r > \frac{1}{n}$. Therefore all open intervals $(0-r,0+r)$ contains a point outside of $A$.
\end{solution}
\end{parts}

\subsection{Theorem 1.3}

\question

\begin{parts}
 
 \part{The union of any collection of open subsets of the real line is also an open subset of the line}
 
 
\begin{solution}
 \\ Let $\bigcup A$ denote the union of subsets $A$. Then
$$x \in \bigcup A \Rightarrow x \in A \Rightarrow x \in (x-r,x+r) \subseteq A \subseteq \bigcup A$$
for some $r > 0$. \\
As all points in $\bigcup A$ are in open intervals which are subsets of $\bigcup A$, the union is open by definition.
\end{solution}

\part{The intersection of any finite collection of open subsets of the real line is also an open subset of the line.}

\begin{solution}
 \\This can be proven with induction. \\
Let $A$ and $B$ be open sets. $x \in A \cap B \Rightarrow x \in A$. Since $A$ is open, $\exists r_A > 0$ such that
$$x \in (x-r_A,x+r_A) \subseteq A$$
The same holds for $B$. Letting $r = \text{min}\{r_A,r_B\}$,
$$x \in (x-r,x+r) \subseteq A \cap B \forall x \in A \cap B$$
The same proof is used for the base case and the induction step.
\end{solution}

\part{Both the empty set and $\mathbb{R}$ itself are open subsets of the real line.}

\begin{solution}
 \\Empty set: \\
 $$a \Rightarrow b$$
 is defined to be true when $a$ is false. The definition of an open set $A$ involves the assumption $x \in A$, which is false, so it is vacuously true for the empty set. \\
 $\mathbb{R}$: \\
 $\forall x \in \mathbb{R}, (x-1,x+1) \in \mathbb{R}$
\end{solution}

\end{parts}

\subsection{Exercise 1.4}

\question{Give an example of an infinite collection of open subsets of the real line whose intersection is not open, thus showing that the finiteness condition in Theorem 1.3(b) is necessary.}

\begin{solution}
 $$\bigcap A \text{ where } A = \{(-r,r)|r>0\}$$
 Obviously $0 \in (-r,r) \forall r > 0$. However, the intersection does not contain any nonzero element, because $\forall x \neq 0$,
 $$x \notin \{-\frac{|x|}{2},\frac{|x|}{2}\}$$
 The open intervals that form the intersection are open sets, but the intersection contains only 1 element, so it is not open.
\end{solution}

\subsection{Theorem 1.6}

\question

\begin{parts}
 
 \part{The intersection of any collection of closed sets is closed.}
 
 
\begin{solution}
 \\Let $C_i$ be a closed set, and the corresponding open set be defined as $O_i = \mathbb{R} - C_i$.
 
 $$\bigcap_{i \in I} C_i = \bigcap_{i \in I} \mathbb{R} - O_i = R - \bigcup_{i \in I} O_i$$
 
 $\bigcup_{i\in I}O_i$ is a union of open sets, so it is open. Hence its complement (intersection of closed sets) is closed.
\end{solution}

\part{The union of any finite collection of closed sets is closed.}

\begin{solution}
 
 $$\bigcup_{n \in \mathbb{N}} C_n =  \bigcup_{n \in \mathbb{N}} \mathbb{R} - O_n = R - \bigcap_{n \in \mathbb{N}} O_n$$
 
 And the proof follows similar to the case above.
\end{solution}

\part{$\emptyset$ and $\mathbb{R}$ itself are both closed}

\begin{solution}
 Their complements are each other, which are open.
\end{solution}
\end{parts}

\subsection{Exercise 1.7}

\question{State precisely what it means when a subset of $\mathbb{R}$ is not closed. (Do this in term of points; saying that a set is not closed if its complement is not open is true, but is not what we want here.}

\begin{solution}
 \\Let that subset be $A$. It is not closed when there is a point outside of it whose every open interval intersects with $A$.
 
 $$\exists x \in \mathbb{R} - A \text{ such that } (x-r,x+r) \cap A \neq \emptyset \forall r > 0$$
\end{solution}

\question{Which of the following subsets of $\mathbb{R}$ are closed? Which are open?}

\begin{parts}
 
 \part{The set $\mathbb{Z}$ of integers.}
 \part{The set of rational numbers.}
 \part{A set consisting of a single point.}
 \part{An interval of the form $[a,b)$, where $a < b$.}
 \part{The set $A = \{x\in\mathbb{R}: x \neq \frac{1}{n} \text{ for } n \in \mathbb{Z}^+\}$.}
 \part{The set $A = \{x\in\mathbb{R}: x \neq \frac{1}{n} \text{ for } n \in \mathbb{Z}^+ \text{ and } x \neq 0\}$.}
 
 
\begin{solution}
 \\Closed: a, c \\
 Open: f
\end{solution}

\end{parts}

\question{Prove that an ordinary closed interval is a closed subset of $\mathbb{R}$, but a closed set need not be a closed interval.}

\begin{solution}
 \\Let the closed interval be $[a,b]$, where $a < b$. We want to prove that its complement is open. Let $x < a$ be in its complement. Then
 $$(x-r,x+r) \text{ where } r = \frac{a-x}{2}$$
 is an open interval in its complement. A similar open interval can be deduced for $x > b$. Therefore, the complement is open, and the closed interval is closed. \\
 For the second part of the question, note that $\mathbb{R}$ is closed, but is not a closed interval. (Or else, let $\mathbb{R} = [a,b]$, and $\mathbb{R}-[a,b] \neq \emptyset$ forms a contradiction.)
\end{solution}

\question{Give an example of an infinite collection of closed subsets of whose union is not closed, thus showing tht the finiteness condition in theorem 1.6(b) is not necessary.}

\begin{solution}
 \\Let the collection be
 $$A = \{[-r,r]|0<r<1\}$$
 It is obvious that $\bigcup A = (-1,1)$. The complement of $(-1,1)$ contains 1, whose every open interval intersects with $(-1,1)$. Hence the complement is not open, so $\bigcup A$ is not closed.
\end{solution}

\subsection{Exercise 2.3}

\question{Show that the absolute value formula, $d(x,y) = |x-y|$ is indeed a metric on the real line. Describe the \textbf{1-ball} centered at 0 in the topology induced by this metric.}

\begin{solution}
Trivial.
\begin{enumerate}
 \item $|x-y| \geq 0 \forall \{x,y\} \subset \mathbb{R}$ \\
 \item $|x-y| = 0$ iff $x=y$ \\
 \item $|x-z|+|y-z|=|x-z|+|z-y| \geq |x-z+z-y| = |x-y|$ \\
 \end{enumerate}
 
 The 1-ball is ordinary closed interval $(-1,1)$.
\end{solution}

\question{Show that the distance formula is a metric on the Euclidean plane. Describe the 1-ball centered at $(0,0)$ in the topology induced by this metric.}

\begin{solution}
 \\In fact, this can be proven for all finite Euclidean spaces $\mathbb{R}^n$, where
 $$d(x,y) = \sqrt{\sum_{i=1}^n (x_i-y_i)^2}$$
 $d(x,y) \geq 0 \forall \{x,y\} \subset \mathbb{R}^n, d(x,y) = 0 \text{ iff } x = y, d(x,y) = d(y,x) \forall \{x,y\} \subset \mathbb{R}^n$ are all trivial. What remains is the triangle inequality.
 Let $a_i = x_i - z_i$ and $b_i = z_i - y_i$. Therefore $a_i + b_i = x_i - y_i$.
 
 
\begin{align*}
 d(x,z) + d(y,z) &= \sqrt{\sum_i a_i^2} + \sqrt{\sum_i b_i^2} \\
 (d(x,z) + d(y,z))^2 &= \sum_i(a_i^2 + b_i^2) + 2\sqrt{\left(\sum_i a_i^2\right)\left(\sum_i b_i^2\right)} \\
 (d(x,z)+d(y,z))^2 - (d(x,y))^2 &= 2\left(\sqrt{\left(\sum_i a_i^2\right)\left(\sum_i b_i^2\right)} - \sum_i a_i b_i\right) \\
\end{align*}
This is greater than 0 (from the Cauchy-Schwarz Inequality). Rearranging and taking the square root of both sides (we can do this because the distance formula is always positive) yields the desired inequality. \\
The 1-ball is the open disk with a radius of 1.
\end{solution}

\question{Define the obvious metric for Euclidean 3-space, $E^3 = \mathbb{R} \times \mathbb{R} \times \mathbb{R} = \{(x,y,z):x,y,z \in \mathbb{R}\}$. Describe the 1-ball centered at the point $(0,0,0)$ in the topology induced by this metric. (In this topology, an $r$ ball really is a ball - hence the name "$r$-ball.")}

\begin{solution}
 The metric is defined in the question above, and the 1-ball is an open sphere with a radius of 1.
\end{solution}

\question{Show that for any metric space $(X,d)$,}

\begin{parts}
 
 \part{The union of any collection of open sets is open.}
 
 
\begin{solution}
 \\Let $O_i$ be open. Then
 $$x \in \bigcup_i O_i \Rightarrow x \in O_i \Rightarrow x \in S_r(x) \subseteq O_i$$
 Since $O_i \subseteq \bigcup O_i$, we have
 $$x \in \bigcup_i O_i \Rightarrow x \in S_r(x) \subseteq \bigcup_i O_i$$
\end{solution}

\part{The intersection of any finite collection of open sets is open.}

\begin{solution}
 \\Similar to the metric case, this can be proven using induction. \\
Let $A$ and $B$ be open sets.
$$x \in A \cap B \Rightarrow x \in A \Rightarrow x \in S_{r_A}(x) \subseteq A$$
Similarly,
$$x \in S_{r_B}(x) \subseteq B$$
Letting $r = \text{min}\{r_A,r_B\}$, we have
$$x \in A \cap B \Rightarrow x \in S_r(x) \subseteq A \cap B$$
\end{solution}

\part{The empty set and $X$ itself are open.}

\begin{solution}
 \\The fact that the empty set is open is vacuously true, and the definition of $S_r(x)$ implies $S_r(x) \subseteq X$, so $X$ is also open.
\end{solution}
\end{parts}

\question{A set can have more than one metric defined on it, and different metrics may give rise to different topologies.}

\begin{parts}
 
 \part{Let $X = \mathbb{R}$ and define a metric on $X$ by $d(x,y) = 1$ if $x \neq y$, $d(x,y) = 0$ if $x = y$. Prove that $d$ is a metric on $X$. What is $S_\frac{1}{2}(0)$ in this metric? Is $(X,d)$ the same space as $\mathbb{R}$ with its usual metric topology? In other words, does this metric give rise to the same topology on $\mathbb{R}$ as the usual metric does?}
 
 
\begin{solution}
 \\The fact that $d$ is a metric is trivial. $S_\frac{1}{2}(0) = \{0\}$. This metric gives rise to a different topology, as $\{0\}$ is open in this topology, but not the usual topology.
\end{solution}

\part{Let $X$ be the Euclidean place and define a metric on $X$ by $d((x_1,x_2),(y_1,y_2)) = |x_1-y_1| + |x_2 - y_2|$. Prove that $d$ is a metric on the plane, and describe the $r$-balls in this metric. Does this metric give rise to the same topology as the usual metric on this plane?}

\begin{solution}
 \\ $d(x,y) \geq 0 \forall \{x,y\} \subset \mathbb{R}^n, d(x,y) = 0 \text{ iff } x = y, d(x,y) = d(y,x) \forall \{x,y\} \subset \mathbb{R}^n$ are all trivial. What remains is the triangle inequality.

\begin{align*}
 d(x,z) + d(y,z) &= |x_1 - z_1| + |x_2 - z_2| + |y_1 - z_1| + |y_2 - z_2| \\
 &= |x_1 - z_1| + |z_1 - y_1| + |x_2 - z_2| + |z_2 - y_2| \\
 &> |x_1 - y_1| + |x_2 - y_2| \\
 &= d(x,y) \\
\end{align*}
The r-balls of $x$ are open squares centered at $x$ with sides of length $r$. It is obvious that all open disks $D$ centered at $x$ contain open squares $S$ also centered at $x$, (such that $x \in S \subset D$) and vice versa. If $A$ is an open set under the usual topology, then
$$x \in A \Rightarrow x \in D$$
Combining this with
$$x \in S \subset D$$
we have
$$x \in A \Rightarrow x \in S$$
so $A$ is an open set under this metric. Similarly, if $B$ is an open set under this metric, $B$ is also an open set under the usual metric. Therefore, both topologies are the same.
\end{solution}
\end{parts}

\subsection{Exercise 3.3}

\question{Let $(\mathbb{Q},d)$ be the space of rational numbers with the metric topology induced by the absolute value metric $d$.}

\begin{parts}
 
 \part{Show that the set $\{y \in \mathbb{Q}:1 < y < 2\}$ is open in $(\mathbb{Q},d)$.}
 
 
\begin{solution}
 \\Denote the set as $A$. Let $r =$ min$\{\frac{2-y}{2},\frac{y-1}{2}\}$. Then
 $$y \in (y-r,y+r) \cap A \subseteq A \forall y \in A$$
\end{solution}

\part{Show that the set $[\sqrt{2},\sqrt{3}]\cap \mathbb{Q}$ is open in $(\mathbb{Q},d)$.}

\begin{solution}
 Noting that both $\sqrt{2}$ and $\sqrt{3}$ are not in the set, the proof can be obtained by replacing the rationals 1 and 2 in the solution above.
\end{solution}
\end{parts}

\question{Let $S$ = [1,2] be given teh topology induced by the absolute value metric. Show that $[1,\frac{3}{2}),(\frac{3}{2},2]$ and $[1,2]$ are all open in $(S,d)$.}

\begin{solution}
 \\For the first set, if $x \neq 1$, let $r =$ min$\{\frac{3}{2}-x,x-1\}$. Then $S_r(x) \cap [1,\frac{3}{2}) = S_r(x)$, and so
 $$x \in [1,\frac{3}{2}) \Rightarrow x \in S_r(x) \subset [1,\frac{3}{2})$$
 If $x = 1$, then $r = \frac{5}{4}$ suffices. \\
 A similar approach is used for the second and third set, noting the special cases for $x=1$ and $x=2$.
\end{solution}

\question{Let $(\mathbb{Z},d)$ be the space of integers with the metric topology induced by the absolute value metric. Show that \textit{every} subset of $\mathbb{Z}$ is open in $(\mathbb{Z},d)$. In particular, any set consisting of a single point is open in $(\mathbb{Z},d)$.}

\begin{solution}
 \\Let $A$ be a subset, and let $r = \frac{1}{2}$. Then $\forall x \in A$,
 $$x \in S_r(x) \cap A = \{x\} \subseteq A$$
\end{solution}

\question{Let $(X,d)$ be a discrete space. Show that every subset of $X$ is open in $(X,d)$.}

\begin{solution}
 \\Let $A$ be a subset. Then
 $$A = \bigcup\{\{x\}|x \in A\}$$
 which is the union of open sets, so $A$ is also open.
\end{solution}

\subsection{Theorem 4.1}

\question{A subset $A$ or $\mathbb{R}$ is open in the metric topology induced by the absolute value metric if and only if for each point $x \in A$, there exist real numbers $a$ and $b$ with $a < x < b$, such that $x \in (a,b) \subseteq A$.}

\begin{solution}
 \\$\Leftarrow$: \\
 Let $r =$ min$\{x-a,b-x\}$. Then $\forall x \in A$,
 $$x \in (x-r,x+r) \subseteq (a,b) \subseteq A$$
 $\Rightarrow$: \\
 As $A$ is open,
 $$x \in A \Rightarrow x \in S_r(x)$$
 so $a = x - r$ and $b = x + r$ suffices.
\end{solution}

\subsection{Exercise 4.3}

\question{Convince yourself that the real line is a totally ordered set. Give an example of a set with an order relation that satisfies (1), (2), and (3) of Definition 4.2, but does \textit{not} satisfy (4). Such an ordered set is called \textbf{partially ordered} since not every pair of elements can be compared. [\textit{Hint}: Consider $\mathcal{P}(X)$ with "$\leq$" thought as "subset of].}

\begin{solution}
 \\Intuitively, real numbers satisfy (1) through (4). A rigorous proof involves Dedekind cuts, which is (probably) out of the scope of this book. \\
 $\subseteq$ in $\mathcal{P}(X)$ is one such example. (1) and (3) are trivial, and (2) is true by definition. However, $\{1,2\}$ and $\{3,4\}$ are elements of $\mathcal{P}(\mathbb{N})$ where both elements cannot be compared.
\end{solution}

\subsection{Exercise 4.5}

\question{Every open subset of $\mathbb{R}$ has cardinality $c$}

\begin{solution}
\\Let $A$ be an open subset. $\forall x \in A$,
$$x \in (x-r,x+r) \subseteq A$$
The function
$$f(y) = \frac{y-x}{r} \times \frac{\pi}{2}$$
is a bijection from $(x-r,x+r)$ to $\mathbb{R}$, so they share the same cardinality $c$. As
$$(x-r,x+r) \subseteq A \subseteq \mathbb{R}$$
$|A| = c$ by the Cantor-Bernstein Theorem.
\end{solution}

\question{For any positive integer $n$, there are open subsets of $[0,\Omega)$ with cardinality $n$. In particular, there are points $x$ in $[0,\Omega)$ such that the singleton set $\{x\}$ is an open set. But $[0,\Omega)$ is not discrete - not every singleton set is open.}

\begin{solution}
 \\$[1,n]$ is a set with cardinality $n$. To show that it is open, $\forall x \in [1,n]$,
 $$0 < x < n+1 \text{ such that } x \in (0,n+1) \subseteq A$$
 Putting $n=1$ gives us a singleton set that is open. However, not all singleton sets are open, e.g. $\{\omega\}$. For it to be open, $a < \omega < b$ such that
 $$\omega \in (a,b) \subseteq \{\omega\}$$
 Since $\omega$ is a limit ordinal, $a < a+1 < \omega < b$, so
 $$a + 1 \in (a,b) - \{\omega\}$$
 which is a contradiction.
 \end{solution}

\question{There are open subsets of $[0,\Omega)$ of cardinality $\aleph_0$ and there are open subsets of $[0,\Omega)$ of cardinality $\aleph_1$.}

\begin{solution}
 \\$\forall n \in \mathbb{N}$, $[n+1,\omega)$ is a subset of cardinality $\aleph_0$. To show that it is open, $\forall x \in [n+1,\omega)$,
 $$n < x < \omega \text{ such that } x \in (n,\omega) \subseteq [n+1,\omega)$$
 Similarly, $[n+1,\Omega)$ is a subset of cardinality $\aleph_1$. The bijective function from $[0,\Omega)$ to $[n+1,\Omega)$ is given by
 $$f(x) = \begin{cases} x + n + 1 & x \in \mathbb{N} \\ x & \text{else} \end{cases}$$
 To show that it is open, $\forall x \in [n+1,\Omega)$,
 $$n < x < \Omega \text{ such that } x \in (n,\Omega) \subseteq [n+1,\Omega)$$
\end{solution}

\question{Supplementary: No countable collection of subsets of $[0,\Omega]$ intersect at $\{\Omega\}$ only.}

\begin{solution}
 \\Let $A_n$ be an open set in the collection. Since $\Omega \in A$,
 $$\Omega \in (a_n,\Omega] \subseteq A$$
 Where $(a_n,\Omega] = [a_n+1,\Omega]$. Let $b_n = a_n+1$, and consider
 $$b = \bigcup_{n \in \mathbb{N}} b_n$$
 The union of ordinals (b) is an ordinal that is greater than all of the ordinals in the union $(b > b_n)$. \\
 Since we have subsets of $A_n$,
 $$\bigcap_{n \in \mathbb{N}} [b_n,\Omega] \subseteq \bigcap_{n \in \mathbb{N}} A_n$$
 Since $b > b_n \forall n \in \mathbb{N}$, $b$ is an element of the set in the left, meaning the countable ordinal $b$ is an element of the intersection, which is a contradiction.
\end{solution}

\subsection{Exercise 4.6}

\question{Show that $\omega = [0,\omega)$ with the order topology is a discrete space, so is the same as $\mathbb{Z}^+\cap\{0\}$ when given the metric topology induced by the absolute value metric on $\mathbb{R}$.}

\begin{solution}
 \\First we prove that all singleton sets $\{x\}$ are open.
 $$x = 0 \Rightarrow x \in [0,1) \subseteq \{0\}$$
 $$x \neq 0 \Rightarrow x \in (x-1,x+1) \subseteq \{x\}$$
 Note that $x-1$ exists $\forall x \neq 0$, as $x$ is a nonzero natural number. \\
 Since
 $$\omega = \mathbb{Z}^+ \cup \{0\}$$
 and both are discrete, they are the same.
\end{solution}

\question{The definition of the order topology is given in terms of open intervals. Let $(X, \leq)$ be a totally ordered set and give it the order topology. Show that a subset of $x$ is open in the order topology if and only if it is a union of open intervals. Because of this, we say that the collection of open intervals is a \textit{basis} for the order topology on a totally ordered set.}

\begin{solution}
 \\Let $A$ be an open subset. Since $A$ is open, $\forall x \in A$,
 $$x \in O_x \subseteq A$$
 where $O_x$ denotes an open interval containing $x$ that is a subset of $A$. Then
 $$A = \bigcup_{x \in A} O_x$$
 So all open sets are unions of open intervals $O_x$. Since open intervals are open sets (this can easily be observed), any union of open intervals are open sets.
\end{solution}

