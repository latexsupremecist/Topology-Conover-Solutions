\section{Some Familiar Topological Spaces and Basic Topological Concepts}

\subsection{Exercises 1.2}


\question{Let $A = (0,1) \cup (1,3)$. For the given $x \in A$, give a value of $r > 0$ such that $(x-r, x+r) \subseteq A$.}

\begin{parts}
 
 \part{$x = \frac{3}{4}$}
 \part{$x = 2$}
 \part{$x = \frac{9}{8}$}
 
\begin{solution}
 $\frac{1}{16}$
\end{solution}

\end{parts}

\question{Prove that A = $(0,1) \cup (1,3)$ is an open subset of $\mathbb{R}$.}

\begin{solution}
\\Case 1: $x \in (0,1)$ \\
Let $d = \text{min}\{1-x,x\}$. $(x-\frac{d}{2},x+\frac{d}{2}) \subseteq A$ is an open set. \\
Case 2: $x \in (1,3)$ \\
Let $d = \text{min}\{3-x,x-1\}$. The proof proceeds similarly as Case 1. \\
As $\forall x \in A \exists$ an open interval $\subseteq A$ which contains $x$, $A$ is an open set by definition.
\end{solution}

\question{Prove than an ordinary open interval is an open subset of $\mathbb{R}$ but that an open set need \textit{not} be an open interval.}

\begin{solution}
 \\All open intervals are open subsets: \\
 Replace 0 with $x-r$ and 1 with $x+r$ as in Case 1 from the question above. \\
 Open sets need not be an open interval: \\
 The empty set is an open set, but no open interval is empty, as it contains $x$ (open intervals are in the form $(x-r,x+r)$).
\end{solution}

\question{State precisely what it means when a subset $A$ of $\mathbb{R}$ is \textit{not} open.}

\begin{solution}
 $\exists x \in A$ such that $(x-r,x+r)-A \neq \emptyset \forall r > 0$
\end{solution}

\question{Prove that the following subsets of $\mathbb{R}$ are not open.}

\begin{parts}
 
 \part{The set of rational numbers}
 
 
\begin{solution}
 \\An open interval is nonempty. As it is an open set, it cannot contain a single point only (next question) so it must contain at least 2 points. WLOG, let the 2 points be $p < q$. Using limits, $\exists n \in \mathbb{N}$ such that $r = p + \frac{\sqrt{2}}{n} < q$. Since all open intervals contain an irrational $r$, no open intervals can be a subset of the rationals, hence it is not open.
\end{solution}

 \part{A set consisting of a single point}
 
 
\begin{solution}
 \\$\forall$ open intervals $(x-r,x+r)$, it contains the distinct points $x$ and $x+\frac{r}{2}$. Therefore, $\{x\}$ cannot be open.
\end{solution}

 \part{An interval of the form $[a,b)$, where $a < b$}
 
 
\begin{solution}
\\ $\forall$ open intervals $(a-r,a+r)$, $b=a-\frac{r}{2}$ is a point in that interval which is outside of $[a,b)$.
\end{solution}

 \part{The set $A = \{x\in\mathbb{R}:x\neq\frac{1}{n},\text{ for }n\in\mathbb{Z}^+\}$}
 
 
\begin{solution}
\\ $\frac{1}{n}$ tends to 0. Therefore $\forall r > 0 \exists n \in \mathbb{N}$ such that $r > \frac{1}{n}$. Therefore all open intervals $(0-r,0+r)$ contains a point outside of $A$.
\end{solution}
\end{parts}

\subsection{Theorem 1.3}

\question

\begin{parts}
 
 \part{The union of any collection of open subsets of the real line is also an open subset of the line}
 
 
\begin{solution}
 \\ Let $\bigcup A$ denote the union of subsets $A$. Then
$$x \in \bigcup A \Rightarrow x \in A \Rightarrow x \in (x-r,x+r) \subseteq A \subseteq \bigcup A$$
for some $r > 0$. \\
As all points in $\bigcup A$ are in open intervals which are subsets of $\bigcup A$, the union is open by definition.
\end{solution}

\part{The intersection of any finite collection of open subsets of the real line is also an open subset of the line.}

\begin{solution}
 \\This can be proven with induction. \\
Let $A$ and $B$ be open sets. $x \in A \cap B \Rightarrow x \in A$. Since $A$ is open, $\exists r_A > 0$ such that
$$x \in (x-r_A,x+r_A) \subseteq A$$
The same holds for $B$. Letting $r = \text{min}\{r_A,r_B\}$,
$$x \in (x-r,x+r) \subseteq A \cap B \forall x \in A \cap B$$
The same proof is used for the base case and the induction step.
\end{solution}

\part{Both the empty set and $\mathbb{R}$ itself are open subsets of the real line.}

\begin{solution}
 \\Empty set: \\
 $$a \Rightarrow b$$
 is defined to be true when $a$ is false. The definition of an open set $A$ involves the assumption $x \in A$, which is false, so it is vacuously true for the empty set. \\
 $\mathbb{R}$: \\
 $\forall x \in \mathbb{R}, (x-1,x+1) \in \mathbb{R}$
\end{solution}

\end{parts}
