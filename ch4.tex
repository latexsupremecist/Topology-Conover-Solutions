\newpage

\section{Topological Spaces and Concepts in General}

\subsection{Exercise 2.6}
\setcounter{question}{0}

\question{Let $X$ be a set. Verify that the indiscrete topology, the discrete topology and the finite-complement topology are in fact topologies on $X$.}

\begin{solution}
 \\Indiscrete Topology: \\
 $\emptyset$ and $X$ are open sets, and any intersection/union of open sets are obviously either empty of $X$. \\
 Discrete Topology: \\
 $\emptyset$ and $X$ are open sets. Since any subset is open, any intersection/union of open sets must be a subset, and hence is open. \\
 Finite-complement Topology: \\
 $\empty set$ and $X$ are open sets. Let $A_i$ denote open sets. Then $X - \bigcup_i A_i \subseteq X - A_i$, where $X - A_i$ has a finite cardinality by definition. Therefore, $X - \bigcup_i A_i$ also has a finite cardinality, so $\bigcup_i A_i$ is open. Now let $B_i$ denote finitely many open sets. $X-B_i$ is finite, and so is $\bigcup_i (X - B_i)$ (finite union of finite sets is finite). Since $\bigcup_i (X - B_i) = X - \bigcap_i B_i$ which is finite, $\bigcap_i B_i$ is an open set.
\end{solution}

\question

\begin{parts}
 
 \part{Verify that Sierpinski space is a topological space.}
 
 
\begin{solution}
It contains $\emptyset$ and $X$. Since there are only 3 open sets, brute forcing through all possible unions/intersections show that they are also open sets.
\end{solution}

\part{We said that there are only three different topologies that can be assigned to the 2 point set $\{0,1\}$.  Is the collection of $\{\emptyset, \{1\},\{0,1\}\}$ one of those three topologies on $\{0,1\}$?}

\begin{solution}
 Yes. The same approach for (a) can be used here, since this is essentially the Sierpinski space with 0 and 1 reversed.
\end{solution}

\part{What is $\{0,1\}$ with the finite-complement topology?}

\begin{solution}
 $\{\emptyset, \{0\}, \{1,\}, \{0,1\}\}$
\end{solution}
\end{parts}

\question{List all topologies that can be assigned to a 3 point set.}

\begin{solution}
 \\$\{\emptyset, \{0,1,2\}\}$ \\
 $\{\emptyset, \{0\}, \{0,1,2\}\}$ \\
 $\{\emptyset, \{1\}, \{0,1,2\}\}$ \\
 $\{\emptyset, \{2\}, \{0,1,2\}\}$ \\
 $\{\emptyset, \{0,1\}, \{0,1,2\}\}$ \\
 $\{\emptyset, \{0,2\}, \{0,1,2\}\}$ \\
 $\{\emptyset, \{1,2\}, \{0,1,2\}\}$ \\
 $\{\emptyset, \{0\}, \{0,1\}, \{0,1,2\}\}$ \\
 $\{\emptyset, \{0\}, \{0,2\}, \{0,1,2\}\}$ \\
 $\{\emptyset, \{1\}, \{0,1\}, \{0,1,2\}\}$ \\
 $\{\emptyset, \{1\}, \{1,2\}, \{0,1,2\}\}$ \\
 $\{\emptyset, \{2\}, \{0,2\}, \{0,1,2\}\}$ \\
 $\{\emptyset, \{2\}, \{1,2\}, \{0,1,2\}\}$ \\
 $\{\emptyset, \{0\}, \{1\}, \{0,1\}, \{0,1,2\}\}$ \\
 $\{\emptyset, \{0\}, \{2\}, \{0,2\}, \{0,1,2\}\}$ \\
 $\{\emptyset, \{1\}, \{2\}, \{1,2\}, \{0,1,2\}\}$ \\
 $\{\emptyset, \{0\}, \{0,1\}, \{0,2\}, \{0,1,2\}\}$ \\
 $\{\emptyset, \{1\}, \{0,1\}, \{1,2\}, \{0,1,2\}\}$ \\
 $\{\emptyset, \{2\}, \{0,2\}, \{1,2\}, \{0,1,2\}\}$ \\
 $\{\emptyset, \{0\}, \{1\}, \{0,1\}, \{0,2\}, \{0,1,2\}\}$ \\
 $\{\emptyset, \{0\}, \{2\}, \{0,1\}, \{0,2\}, \{0,1,2\}\}$ \\
 $\{\emptyset, \{0\}, \{1\}, \{0,1\}, \{1,2\}, \{0,1,2\}\}$ \\
 $\{\emptyset, \{1\}, \{2\}, \{0,1\}, \{1,2\}, \{0,1,2\}\}$ \\
 $\{\emptyset, \{0\}, \{2\}, \{0,2\}, \{1,2\}, \{0,1,2\}\}$ \\
 $\{\emptyset, \{1\}, \{2\}, \{0,2\}, \{1,2\}, \{0,1,2\}\}$ \\
 $\{\emptyset, \{0\}, \{1\}, \{2\}, \{0,1\}, \{0,2\}, \{1,2\}, \{0,1,2\}\}$ \\
\end{solution}

\question{Verify that the Sorgenfrey topology defined on the real line is in fact a topology. Is the interval $(0,1)$ open in this topology? How about $(0,1]$? Is $[0,1]$ closed?}

\begin{solution}
 \\The Sorgenfrey topology obviously contains $\emptyset$ and $\R$. Let $A_i$ denote open sets. Then
 $$\forall x \in \bigcup_i A_i, x \in A_i \Rightarrow x \in [a,b) \subseteq A_i \subseteq \bigcup_i A_i$$
 so any union of open sets is open. Now let $B_i$ denote finitely many open sets. Now if $x \in B_i$, we let $\{p_i,q_i\} \subset \R$ such that $x \in [p_i,q_i) \subseteq B_i$. Let $P = \{p_i\}$ and $Q = \{q_i\}$. Since both $P$ and $Q$ are finite, $P$ has a maximum $p$ and $Q$ has a minimum $Q$.
 Now
 $$x \in \bigcap_i B_i \Rightarrow x \in [p,q) \subseteq [p_i,q_i) \subseteq B_i \forall i$$
 Since $[p,q)$ is a subset of $B_i \forall i$, it is a subset of $\bigcap_i B_i$, hence any finite intersection of open sets is open. This shows that the Sorgenfrey topology is a topology. \\
 $(0,1)$ is open, because
 $$\forall x \in (0,1), x \in [x,1) \subset (0,1)$$
 $(0,1]$ is not open, because $1 \in (0,1]$, but if $1 \in [a,b)$, then $\frac{1+b}{2}$ is an element of $[a,b)$ but not $(0,1]$, so $1 \in [a,b) \subseteq (0,1]$ cannot be true. \\
 Consider $A = \R - [0,1]$ and let $x \in A$. Either $x < 0$, and $x \in [x,\frac{x}{2}) \subset A$, or $x > 1$, and $x \in [x,x+1)$. Therefore, $A$ is open, so $[0,1]$ is closed.
\end{solution}

\question{Consider the topological spaces $(\R, \mathcal{I}), (\R, \mathcal{D}), (\R, \mathcal{U}), (\R, \mathcal{S})$ and with the finite-complement topology (where $\mathcal{U}$ denotes the usual topology on $\R$ s in Chapter 3).}

\begin{parts}
 
 \part{If $p \in \R$, is $\{p\}$ open in any of these spaces? Which ones?}
 
 
\begin{solution}
 $(\R, \mathcal{D})$. 
\end{solution}

\part{If $p \in \R$, is $\{p\}$ closed in any of these spaces? Which ones?}

\begin{solution}
$(\R, \mathcal{D}), (\R, \mathcal{U}), (\R, \mathcal{S})$, and the finite-complement topology.
\end{solution}

\part{In which of these spaces is $(a,b)$ open? $[a,b)$? $(a,b]$? $[a,b]$?}

\begin{solution}
 \\$(a,b)$: \\
 $(\R, \mathcal{D}), (\R, \mathcal{U}), (\R, \mathcal{S})$ \\
 $[a,b)$: \\
 $(\R, \mathcal{D}), (\R, \mathcal{S})$ \\
 $(a,b]$: \\
 $(\R, \mathcal{D})$ \\
 $[a,b]$: \\
 $(\R, \mathcal{D})$ \\
\end{solution}

\part{Is the set $\{x \in \R: x \neq \frac{1}{n}\}$ open in any of the spaces? Is it closed in any of them?}

\begin{solution}
 \\Open: \\
 $(\R, \mathcal{D})$ \\
 Closed: \\
 $(\R, \mathcal{D})$ \\
\end{solution}

\part{Is the set $\{x \in \R: x \neq \frac{1}{n} \text{ and } x \neq 0\}$ open in any of the spaces? Is it closed in any of them?}

\begin{solution}
 \\Open: \\
 $(\R, \mathcal{D})$ \\
 Closed: \\
 $(\R, \mathcal{D}), (\R, \mathcal{U}), (\R, \mathcal{S})$ \\
\end{solution}

\end{parts}

\question{Consider the spaces of Problem 5 above again, together with the three spaces that can be defined on $\{0,1\}$.}

\begin{parts}
 
 \part{In which of these spaces are true: If $x$ and $y$ are two distinct points in the space then either there exists an open set $U$ such that $x \in U$ and $y \notin U$, or there exists an open set $V$ such that $y \in V$ and $x \notin V$. (A space for which this statement holds is called a T$_0$-space.)}
 
\begin{solution}
 $(\R, \mathcal{D}), (\R, \mathcal{U}), (\R, \mathcal{S})$, the finite-complement topology, and the Sierpinski space.
\end{solution}

\part{In which of these spaces is the following statement true: If $x$ and $Y$ are two distinct points in the space, then there exists an open set $U$ such that $x \in U$ and $y \notin U$, and there exists an open set $v$ such that $y \in V$ and $x \notin V$. (A space for which this statement holds is called a T$_1$-space.)}

\begin{solution}
 $(\R, \mathcal{D}), (\R, \mathcal{U}), (\R, \mathcal{S})$, and the finite-complement topology.
\end{solution}

\part{In which of these spaces is the following statement true: If $x$ and $y$ are two distinct points in the space, the there exist open sets $U$ and $V$ such that $x \in U$, $y \in V$ and $U \cap V = \emptyset$. (A space for which this statement holds is called a T$_2$-space or a Hausdorff space.)}

\begin{solution}
  $(\R, \mathcal{D}), (\R, \mathcal{U}), (\R, \mathcal{S})$
\end{solution}

\end{parts}

\question{Show that every T$_2$-space is a T$_1$-space, and that every T$_1$-space is a T$_0$-space, and give an example of a T$_0$-space that is not a T$_1$-space, and an example of a T$_1$-space that is not a T$_2$-space.}

\begin{solution}
 \\T$_2$-space $\Rightarrow$ T$_1$-space: Since $U \cap V \neq \emptyset$, $x \notin V$ and $y \notin U$, so $V$ and $U$ are open sets that make the space a T$_1$-space. \\
T$_1$-space $\Rightarrow$ T$_0$-space: either one of $U$ and $V$ make the space a T$_0$-space. \\
T$_0$-space that is not a T$_1$-space: the Sierpinski space \\
T$_1$-space that is not a T$_2$-space: the finite-complement topology \\
\end{solution}

\question{A topological space $X$ is said to be \textbf{metrizable} if a metric can be defined on $X$ so that a set is open in the metric topology induced by this metric if and only if it is open in the topology that is already on the space.}

\begin{parts}
 
 \part{Let $X$ be a set with more than one point. Prove that $(X, \mathcal{I})$ is \textit{not} metrizable. Thus the indiscrete topology on a set with more than one point is an example of a topological space that is \textit{not} a metric space.}
 
 
\begin{solution}
 \\If $X$ has more than one point, it has 2 distinct points $x$ and $y$, where we let $r = d(x,y) > 0$ by the definition of a metric. $S_{\frac{r}{2}}(x)$ is an open space according to the metric. However, it is neither empty (contains $x$) nor the universe (does not contain $y$). This forms a contradiction.
\end{solution}

\part{Let $X$ be a set. Define a function from $X \times X = \{(x,y):x,y\in X\}$ to $\R$ by $$d(x,y) = \begin{cases} 1 & \text{if } x \neq y \\ 0 \text{if } x = y \end{cases}$$ Prove that $d$ is a metric on $X$. What is the metric topology induced by $d$?}

\begin{solution}
 \\Obviously, $d(x,y) = 0$ if and only if $x=y$, $d(x,y) \geq 0$, $d(x,y) = d(y,x)$. For the triangle inequality $d(x,y) \leq d(x,z) + d(y,z)$, note that it is trivial if $x=y$. If not, at least one of $d(x,z)$ and $d(y,z)$ must be nonzero, so the inequality holds. Therefore, $d(x,y)$ is a metric. \\
 $\forall x \in X, S_{0.5}(x) = \{x\}$. Since all singleton sets are open, the metric topology induced by $d$ is the discrete topology.
\end{solution}
\end{parts}

\subsection{Theorem 3.2} \label{thm4.3.2}
\setcounter{question}{0}

\question{Let $(X, d_1)$ and $(Y, d_2)$ be metric spaces and let $f: X \rightarrow Y$. Then $f$ is continuous at $x_0 \in X$ if and only if whenever $V$ is an open subset of $Y$ with $f(x_0) \in V$, then there exists an open subset $U$ of $X$ such that $x_0 \in U$ and $f(U) \subseteq V$.}
 
 
\begin{solution}
 \\$\Rightarrow$: \\
 Let $V$ be open. Since $V$ is open, $\forall v = f(x_0) \in V$, $\exists \epsilon > 0$ where $S_\epsilon(v) \subseteq V$. Since continuity is implied, $\exists \delta > 0$ where $f(S_\delta(x_0)) \subseteq S_\epsilon(v)$. Therefore $S_\delta(x_0)$ is the desired open $U$. \\
 $\Leftarrow$: \\
 Let $f(x_0) = v$. $\forall \epsilon > 0$, $V = S_\epsilon(v)$ is open. Then an open $U$ exists where $x_0 \in U$. By definition, U is open, so $\exists \delta > 0$ such that $S_\delta(x_0) \in U$. Then
 $$f(S_\delta(x_0)) \subseteq f(U) \subseteq V = S_\epsilon(v)$$
 This demonstrates that $f$ is continuous by the $\epsilon-\delta$ definition.
\end{solution}

\subsection{Theorem 3.4}
\setcounter{question}{0}

\question{Let $X$ and $Y$ be topological spaces and let $f:X \rightarrow Y$. Then $f$ is continuous on $X$ if and only if whenever $V$ is an open subset of $Y$, then $f^{-1}(V)$ is open in $X$.}

\begin{solution}
 \\$\Rightarrow$: \\
 Based on \hyperref[thm4.3.2]{Theorem 3.2}, we have an open $U_x \forall f(x) \in V$ such that $f(U_x) \subseteq V$. Therefore, $U_x \subseteq f^{-1}(V)$. Then
 $$f^{-1}(V) = \bigcup_{x \in f^{-1}(V)} U_x$$
 Since $f^{-1}(V)$ is a union of open sets, it is open. \\
 $\Leftarrow$: \\
 Similar to the second part of \hyperref[thm4.3.2]{Theorem 3.2}, let $f(x_0) = v$. $\forall \epsilon > 0$, $V = S_\epsilon(v)$ is open. Then $f^{-1}(V)$ is open (and contains $x_0$), so $\exists \delta > 0$ such that $S_\delta(x_0) \in f^{-1}(V)$. Then
 $$f(S_\delta(x_0)) \subseteq V = S_\epsilon(v)$$
 This demonstrates that $f$ is continuous by the $\epsilon-\delta$ definition.
\end{solution}

\subsection{Exercise 3.5}
\setcounter{question}{0}

\question{Consider $(\R, \mathcal{I}), (\R, \mathcal{D}), (\R, \mathcal{U}), (\R, \mathcal{S})$ and $(\R, \mathcal{F})$, the real line with the indiscrete topology, the discrete topology, the usual metric topology, the Sorgenfrey topology and the finite-complement topology, respectively. Let $f: \R \rightarrow \R$ be the identity function defined by $f(r)=r$ for all real numbers $r$. Determine all possible choices for $\mathcal{J}_1$ and $\mathcal{J}_2$ from $\mathcal{I}, \mathcal{D}, \mathcal{U}, \mathcal{S}, \mathcal{F}$ so that $f:(\R, \mathcal{J}_1) \rightarrow (\R, \mathcal{J}_2)$ is continuous.}

\begin{solution}
 
 \begin{center}
\begin{tabular}{|c||c|c|c|c|c|}
\hline
 & $\mathcal{I}$ & $\mathcal{D}$ & $\mathcal{U}$ & $\mathcal{S}$ & $\mathcal{F}$ \\
 \hline \hline
 $\mathcal{I}$ & \cmark & \xmark & \xmark & \xmark & \xmark \\
 \hline
 $\mathcal{D}$ & \cmark & \cmark & \cmark & \cmark & \cmark \\
 \hline
 $\mathcal{U}$ & \cmark & \xmark & \cmark & \xmark & \cmark \\
 \hline
 $\mathcal{S}$ & \cmark & \xmark & \cmark & \cmark & \cmark \\
 \hline
 $\mathcal{F}$ & \cmark & \xmark & \xmark & \xmark & \cmark \\
 \hline
\end{tabular}
\end{center}
 Where the rows denote $\mathcal{J}_1$ and the columns denote $\mathcal{J}_2$. 
\end{solution}

\question{Let $x$ be a set and let $\mathcal{J}$ be any topology on $X$. There is a topology $\mathcal{J}$' that can be assigned to $X$ so that the identity function from $(X, \mathcal{J}')$ to $(X\mathcal{J})$ is always continuous no matter what $\mathcal{J}$ s. What is it?}

\begin{solution}
 The discrete topology $\mathcal{D}$.
\end{solution}

\question{Let $X$ be any set and let $\mathcal{J}$ be any topology on $X$. There is a topology $\mathcal{J}'$ that can be given to $X$ so that the identity function from $(X, \mathcal{J})$ to $(X, \mathcal{J}')$ is always continuous no matter what $\mathcal{J}$ is. What is it?}

\begin{solution}
 The indiscrete topology $\mathcal{I}$.
\end{solution}

\question{A function that preserves open sets is called an \textbf{open function}. More precisely, a function $f:X \rightarrow Y$ is an open function if whenever $U$ is open in $X$, then $f(U)$ is open in $Y$.}

\begin{parts}
 
 \part{Give an example of a continuous function that is not open.}
 
\begin{solution}
 $f(r) = r$ where $X$ is the real line with the discrete topology and $Y$ is the real line with the indiscrete topology.
\end{solution}

\part{Give an example of an open function that is not continuous.}

\begin{solution}
 $f(r) = r$ where $X$ is the real line with the indiscrete topology and $Y$ is the real line with the discrete topology.
\end{solution}

\end{parts}

\subsection{Exercise 4.2}
\setcounter{question}{0}

\question{Consider the spaces $(\R, \mathcal{I}), (\R, \mathcal{D}), (\R, \mathcal{U}), (\R, \mathcal{S})$ and $(\R, \mathcal{F})$. In which of these spaces is:}

\begin{parts}
 
 \part{$(0,2)$ a neighbourhood of 1?}
 
 \part{$[0,2]$ a neighbourhood of 1?}
 
 \part{$[0,2]$ a neighbourhood of 0?}
 
 \part{\{0\} a neighbourhood of 0?}
 
 
\begin{solution}
 \\(a): $\mathcal{D}, \mathcal{U}, \mathcal{S}$ \\
 (b): $\mathcal{D}, \mathcal{U}, \mathcal{S}$ \\
 (c): $\mathcal{D}, \mathcal{S}$ \\
 (d): $\mathcal{D}$ \\
\end{solution}
\end{parts}

\question{In the plane with its usual topology, is the unit square $$\{(x,y):0 \leq x \leq 1, 0 \leq y \leq 1\}$$ a neighbourhood of any point in it? Which points?}

\begin{solution}
 \\Any point that is not the boundary, or more explicitly,
 $$\{(x,y):0 < x < 1, 0 < y < 1\}$$
\end{solution}

\subsection{Theorem 4.3}
\setcounter{question}{0}

\question{Let $X$ be a topological space. Then $U \subseteq X$ is open if and only if $U$ is a neighbourhood of each point $x \in U$.}

\begin{solution}
 \\$\Rightarrow$: \\
 Since $U$ is open, $x \in U$ implies $x \in O \subseteq U$ for some open set $O$. Since $U$ is a superset of an open $O$ that contains $x$, $U$ is a neighbourhood of $x$ by definition. \\
 $\Leftarrow$: \\
 Since $U$ is a neighbourhood, $x \in U$ implies $x \in O \subseteq U$ where $O$ is some open set. This means $U$ is open by definition.
\end{solution}

\subsection{Exercise 4.5}
\setcounter{question}{0}

\question{Let $X$ be the real line with the indiscrete topology and consider the sequence $\{\frac{1}{n}: n \in \Z^+\}$. Prove that if $r$ is any point of $X$, then this sequence converges to $r$.}

\begin{solution}
 \\Let $O$ be an open set containing $r$. Since this is the indiscrete topology, $O=X$ is the only possibility. Then $\forall n > 0$, $f(n) \in O$, so the sequence converges to $r$.
\end{solution}

\question{Let $X = \R$ with the finite-complement topology. Prove that $\{\frac{1}{n}:n \in \Z^+\}$ converges to every point in the space.}

\begin{solution}
 \\Let $O$ be an open set containing $r$. Since this is the finite-complement topology, $X-O$ is finite. Let $A = \{a \in X-O|a > 0\}$. If $A$ is empty, let $N$ = 0. Or else, $A$ must have a minimum $m$ that is a positive number. Since $\frac{1}{n}$ tends to 0, $\exists N \in \N$ such that $\frac{1}{n} < m \forall n > N$. Now $\forall n > N$, $f(n) \in O$, so the sequence tends to $r$.
\end{solution}

\question{Recall that a topological space $X$ is a Hausdorff space if distinct points of $X$ are contained in disjoint open sets.}

\begin{parts}
 
 \part{Prove that a space is a Hausdorff space if and only if distinct points are contained in disjoint neighbourhoods.}
 
\begin{solution}
 \\$\Rightarrow$: \\
 Open sets are also neighbourhoods, so distinct points in the space are contained in disjoint neighbourhoods. \\
 $\Leftarrow$: \\
 Let the distinct points be $x$ and $y$, and the disjoint neighbourhoods $N_x$ and $N_y$. Since $N_x$ is a neighbourhoods of $x$, $x \in O_x \subseteq N_x$, and the same holds for $y$. Then $x$ and $y$ are in disjoint open sets $O_x$ and $O_y$, so the space is a Hausdorff space.
\end{solution}

\part{Prove that every metric space is a Hausdorff space.}

\begin{solution}
\\Let $x$ and $y$ be two distinct points, and let $R = d(x,y), r = \frac{R}{2}$. Consider a point $p \in S_r(x)$. Rearranging the triangle inequality,
$$d(p,y) \geq R - d(p,x) > R - r = r$$
so $p$ cannot be in $S_r(y)$. The same holds if $p \in S_r(y)$. This means for all distinct $x, y$, there exists disjoint open sets $S_r(x)$ and $S_r(y)$ which contain $x$ and $y$ respectively, making it a Hausdorff space.
\end{solution}

\part{Prove that in a Hausdorff space, if a sequence converges, then it converges to exactly one point. Deduced that the real line with the finite complement topology is not a Hausdorff space.}

\begin{solution}
 \\Proof by contradiction. Let the sequence converge to distinct points $x$ and $y$. Let $O_x$ and $O_y$ be a pair of disjoint open sets that contain $x$ and $y$ respectively. Since the sequence converges to $x$, $\exists N_x \in N$ such that $n > N \Rightarrow f(n) \in O_x$. There exists a similar $N_y$. Letting $N$ = max$\{N_x, N_y\}$, then $\forall n > N, x_n$ is in both $O_x$ and $O_y$ at the same time, which is a contradiction as both sets are disjoint.
\end{solution}

\end{parts}

\question{Consider $X = [0, \Omega]$ with the order topology as defined in Section 4 of Chapter 3. We proved that in a metric space, a set $S$ is closed if and only if whenever a sequence of points of $S$ converges to a point $x$ in the space, then $x \in S$.}

\begin{parts}
 
 \part{Prove, usig the definition of closed set, that $S = [0, \Omega)$ is not a closed subset of $X = [0, \Omega]$ with the order topology.}
 
\begin{solution}
 \\If it is a closed set, then $\{\Omega\}$ must be open in the order topology, so
 $$\exists a \in X: \Omega \in (a,\Omega] \subseteq \{\Omega\}$$
For the interval to make sense, $a < \Omega$. Noting that $\Omega$ is a limit ordinal,
$$a < \Omega \Rightarrow a+1 < \Omega$$
so $a+1 \neq \Omega$ is an element of $(a, \Omega]$, which is a contradiction, since its superset $\{\Omega\}$ does not contain $a+1$.
 \end{solution}
 
 \part{Prove that if a sequence of points of $S = [0, \Omega)$ converges to a point $x \in [0, \Omega]$, then $x \in S$.}
 
\begin{solution}
\\The sequence $x_n$ cannot tend to $\Omega$ if it has a finite length, since it then as a maximum $m$, and $x_n \notin (m, \Omega] \forall n \in \N$. and If not, consider
 $$l = \bigcap_{n \in \N} (x_n, \Omega]$$
 Obviously $\Omega \in l$, but if $y \neq \Omega \in l$, then
 $$y > x_n \forall n \in N \Rightarrow x_n \notin [y, \Omega] \forall n \in N$$
 Loosely speaking, $x_n$ can never reach $y$, so it can never "enter" the open set $[y, \Omega]$. However, such a $y$ always exists by \hyperref[supp]{this}.
\end{solution}

\part{Deduce that $X = [0, \Omega]$ with the order topology is not a metric space, but}

\begin{solution}
 \\If it is a metric space, then consider
 $$A = \bigcap_{n \in \N} S_{\frac{1}{n}}(\Omega)$$
 By \hyperref[supp]{this}, $\exists a \in A: a \neq \Omega$. Let $d(a, \Omega) = r$. Since $\frac{1}{n}$ tends to 0, we know that $r > \frac{1}{i}$ for some $i \in \N$. Then $a \notin S_{\frac{1}{i}}(\Omega)$, so $a \notin A$, which is a contradiction.
\end{solution}

\part{Prove that $X = [0, \Omega]$ with the order topology is a Hausdorff space.}

\begin{solution}
 Let $a < b$ be two distinct points. Then $[0, a]$ and $(a, \Omega]$ are disjoint open sets that contain $a$ and $b$ respectively.
\end{solution}

\end{parts}

\subsection{Theorem 4.6}
\setcounter{question}{0}

\question{Let $X$ and $Y$ be topological spaces and let $f: X \rightarrow Y$. Then the funtion $f$ is continuous at a point $x_0 \in X$ if and only if for every neighbourhood $N_2$ of $f(x_0)$ in $Y$, there is a neighbourhood $N_1$ of $x_0$ in $X$ such that $f(N_1) \subseteq N_2$.}

\begin{solution}
 \\$\Rightarrow$: \\
 If $N_2$ is a neighbourhood, then $\exists$ an open $O_2 \subseteq N_2$ which contains $f(x_0)$. Since $f$ is continuous, $f^{-1}(O_2)$ is open, and acts as the desired $N_1$. \\
 $\Leftarrow$: \\
 Let $O_2$ be an open set containing $f(x_0)$. Then there exists a $N_1$, and $x_0 \in O_1 \subseteq N_1$ for some open $O_1$ by the definition of a neighbourhood. Since
 $$f(O_1) \subseteq O_2$$
 $f$ is continuous by definition.
\end{solution}

\section{Closed Sets and Closure}
\subsection{Theorem 5.2}
\setcounter{question}{0}

\question{Let $X$ be a topological space. Then}

\begin{enumerate}
 \item $X$ and $\emptyset$ are closed. \\
 \item The intersection of an arbitrary collection of closed sets is closed. \\
 \item The union of a finite collection of closed sets is closed. \\
\end{enumerate}

\begin{solution}
 \\Since the complements of $X$ and $\emptyset$ are each other (which are open), they are also closed. \\
 Let $A$ be a set where $a \in A$ is closed. Then
 $$X - \bigcap A = \bigcup_{a \in A} X-a$$
 which is open. So the intersection of closed sets is closed. Now let $B$ be a finite set where $b \in B$ is closed. Similarly,
 $$X - \bigcup B = \bigcap_{b \in B} X-b$$
 which is open, so a finite union of closed subsets is closed.
\end{solution}

\subsection{Exercise 5.3}
\setcounter{question}{0}

\question{Consider again the spaces $(\R, \mathbb{I}), (\R, \mathbb{D}), (\R, \mathbb{U}), (\R, \mathbb{S}), (\R, \mathbb{F})$. In which of these spaces is:}

\begin{parts}
 \part{$[0,1]$ closed?}
 \part{$(0,1)$ closed?}
 \part{$[0,1)$ closed?}
 \part{$(0,1]$ closed?}
 \part{$\{0\}$ closed?}
\end{parts}

\begin{solution}
 \\(a): $\mathbb{D}, \mathbb{U}, \mathbb{S}$ \\
 (b): $\mathbb{D}$ \\
 (c): $\mathbb{D}, \mathbb{S}$ \\
 (d): $\mathbb{D}$ \\
 (e): $\mathbb{D}, \mathbb{U}, \mathbb{S}, \mathbb{F}$ \\
\end{solution}

\question{Recall that a space is a T$_1$-space if for every pair of distinct points $x$ and $y$ in the space, there is an open set $U$ such that $x \in V$ and $y \notin U$, and there is an open set $V$ such that $y \in V$ and $x \notin V$.}

\begin{parts}
 \part{Restate the definition of a T$_1$-space in terms of neighbourhoods.}
 
 
\begin{solution}
 A space is a T$_1$-space if and only if for every distinct $x, y$ pair, there exists a neighbourhood of $x$ that does not contain $y$, and there exists a neighbourhood of $y$ that does not contain $x$.
\end{solution}

\part{Prove that a space is a T$_1$-space if and only if for each point $p$ in the space, the singleton set $\{p\}$ is a closed set.}

\begin{solution}
 \\$\Rightarrow$: \\
 Let the space be $X$. $\forall q \neq p$, let $O_q$ be an open set that contains $q$ but not $p$. Such a set always exists because the space is T$_1$. Therefore $\bigcup_{q \neq p \in X} O_q = X - \{p\}$ is open, so the singleton set $\{p\}$ is closed. \\
 $\Leftarrow$: \\
 Let the space be $X$, and $x, y$ be a pair of distinct points. Then $X - \{x\}$ and $X - \{y\}$ are open sets which contain exactly one of $x, y$ respectively.
\end{solution}
\end{parts}

\subsection{Theorem 5.4}
\setcounter{question}{0}

\question{Let $X$ and $Y$ be topological spaces and let $f: X \rightarrow Y$. Then $f$ is continuous (on $X$) if whenever $F$ is a closed set in $Y$, then $f^{-1}(F)$ is closed in $X$.}

\begin{solution}
 \\Since open sets and closed sets are complements, $f^{-1}$ preserving closed sets it equivalent to it preserving open sets. \\
 $\Rightarrow$: \\
 Let $F$ be an closed set in $Y$. Let $G = Y - F$, where $G$ is open. Since $f$ is continuous, $f^{-1}(G)$ is also open. Then $f^{-1}(F) = X - f^{-1}(G)$ is closed. \\
 $\Leftarrow$: \\
 Let $G$ be an open set in $Y$. Let $F = Y - G$, where $F$ is closed. Since $f^{-1}(F)$ is closed, $f^{-1}(G) = X - f^{-1}(F)$ is open. \\
\end{solution}

\subsection{Exercise 5.5}
\setcounter{question}{0}

A function $f: X \rightarrow Y$ is a closed function if whenever $F$ is closed in $X$ then $f(F)$ is closed in $Y$.

\question{Give an example of a closed function that is not continuous.}

\begin{solution}
 $f(r) = r$ from $(\R, \mathbb{U})$ to $(\R, \mathbb{D})$ is a closed function that is not continuous.
\end{solution}

\question{Give an example of a continuous function that is not closed.}

\begin{solution}
 $f(r) = r$ from $(\R, \mathbb{U})$ to $(\R, \mathbb{I})$ is a continuous function that is not closed.
\end{solution}

\subsection{Theorem 5.7}
\setcounter{question}{0}

The closure of a set $A$ is defined as
$$\overline{A} = \bigcap \{F:F \text{ is closed in } X \text{ and } A \subseteq F\}$$

\question{Let $X$ be a topological space and let $A \subseteq X$. Then}

\begin{parts}
 \part{$\overline{A}$ is a closed set.}
 
 
\begin{solution}
 $\overline{A}$ is an intersection of closed sets, so it too is closed.
\end{solution}

\part{$A \subseteq \overline{A}$}

\begin{solution}
 $x \in A \Rightarrow x \in F \forall F \iff x \in \overline{A}$
\end{solution}

\part{$\overline{A}$ is the smallest closed set that contains $A$ in the sense than if $S$ is closed and $A \subseteq S$, then $\overline{A} \subseteq S$ also.}

\begin{solution}
 Such an $S$ corresponds to $F$ as stated above, so $x \in S \Rightarrow x \in S$ by definition.
\end{solution}
\end{parts}

\subsection{Theorem 5.8} \label{thm4.5.8}
\setcounter{question}{0}

\question{Let $X$ be a topological space and let $A \subseteq X$. Then if $x \in X, x \in \overline{A}$ if and only if $N \cap A \neq \emptyset$ for all neighbourhoods $N$ of $x$.}

\begin{solution}
 \\$\Rightarrow$: \\
 Proof by contradiction: if there exists a neighbourhood $N$ of $x$ which is disjoint with $A$, then there exists an open set $O_x$ containing $x$ that is disjoint with $A$, and $C_x = X - O_x$ is a closed set containing $A$ that does not contain $x$. Then $x \in \overline{A} \Rightarrow x \in C_x$, which is a contradiction. \\
 $\Leftarrow$: \\
 Proof by contradiction: let $x \notin \overline{A}$. Then there exists a closed set $C_x \in X$ that contains $A$ but not $x$. Then $O_x = X - C_x$ is an open set containing $x$ that is disjoint with $A$. Since all open sets are neighbourhoods, this means there exists a neighbourhood of $x$ that is disjoint with $A$, which contradicts the assumption.
\end{solution}

\subsection{Exercise 5.9}
\setcounter{question}{0}

\question{In the spaces on the real line obtained by giving it the indiscrete topology, the discrete topology, the usual metric topology, the Sorgenfrey topology, and the finite-complement topology, what is $\overline{A}$ if $A$ is}

\begin{parts}
 \part{$(0,1)$.}
 
\begin{solution}
 \\$\mathbb{I}: \R$ \\
 $\mathbb{D}: (0,1)$ \\
 $\mathbb{U}: [0,1]$ \\
 $\mathbb{S}: [0,1)$ \\
 $\mathbb{F}: \R$ \\
\end{solution}

 \part{$[0,1]$.}
 
 \begin{solution}
 \\$\mathbb{I}: \R$ \\
 $\mathbb{D}: [0,1]$ \\
 $\mathbb{U}: [0,1]$ \\
 $\mathbb{S}: [0,1]$ \\
 $\mathbb{F}: \R$ \\
\end{solution}

 \part{$[0,1)$.}
 
 \begin{solution}
 \\$\mathbb{I}: \R$ \\
 $\mathbb{D}: [0,1)$ \\
 $\mathbb{U}: [0,1]$ \\
 $\mathbb{S}: [0,1)$ \\
 $\mathbb{F}: \R$ \\
\end{solution}

 \part{$(0,1]$.}
 
 \begin{solution}
 \\$\mathbb{I}: \R$ \\
 $\mathbb{D}: (0,1]$ \\
 $\mathbb{U}: [0,1]$ \\
 $\mathbb{S}: [0,1]$ \\
 $\mathbb{F}: \R$ \\
\end{solution}

 \part{$\{0\}$.}
 
 \begin{solution}
 \\$\mathbb{I}: \R$ \\
 $\mathbb{D}: \{0\}$ \\
 $\mathbb{U}: \{0\}$ \\
 $\mathbb{S}: \{0\}$ \\
 $\mathbb{F}: \{0\}$ \\
\end{solution}

 \part{$\Q$}
 
  \begin{solution}
 \\$\mathbb{I}: \R$ \\
 $\mathbb{D}: \Q$ \\
 $\mathbb{U}: \R$ \\
 $\mathbb{S}: \R$ \\
 $\mathbb{F}: \R$ \\
\end{solution}

 \part{$\{\frac{1}{n}:n \in \Z^+\}$}
 
  \begin{solution}
 \\$\mathbb{I}: \R$ \\
 $\mathbb{D}: \{\frac{1}{n}:n \in \Z^+\}$ \\
 $\mathbb{U}: \{\frac{1}{n}:n \in \Z^+\} \cup \{0\}$ \\
 $\mathbb{S}: \{\frac{1}{n}:n \in \Z^+\} \cup \{0\}$ \\
 $\mathbb{F}: \R$ \\
\end{solution}

 \part{$\emptyset$}
 
   \begin{solution}
 \\$\mathbb{I}: \emptyset$ \\
 $\mathbb{D}: \emptyset$ \\
 $\mathbb{U}: \emptyset$ \\
 $\mathbb{S}: \emptyset$ \\
 $\mathbb{F}: \emptyset$ \\
\end{solution}

\end{parts}

\question{Instead of giving you a specific problem, this exercise asks you to formulate the problem and then to solve them.}

\begin{parts}
 \part{How does closure behave with respect to unions?}
 
 
\begin{solution}
  \\A finite union of closures is equal to a closure of finite unions, or
 $$\overline{\bigcup_{i \in I} A_i} = \bigcup_{i \in I} \overline{A_i}$$
 where $I$ is finite. \\
 LHS $\subseteq$ RHS: \\
 Proof by contradiction. Let $x \in$ LHS but $x \notin$ RHS. Let $O_i$ be an open set containing $x$ but disjoint with $A_i$. Then $\bigcap_{i \in I} O_i$ is open, contains $x$, but disjoint with $\bigcup_{i \in I} A_i$, meaning $x \notin$ LHS, which is a contradiction. \\
 RHS $\subseteq$ LHS: \\
 Let $x \in RHS$, and $N$ be an arbitrary neighbourhood of $x$. Then $N \cup \bigcup_{i \in } A_i$ is nonempty, so $x \in LHS$. This also shows that RHS $\subseteq$ LHS even when $I$ is not finite.
\end{solution}

 \part{How does closure behave with respect to intersections?}
 
\begin{solution}
 \\A closure of intersections is a subset of an intersection of closures, or
 $$\overline{\bigcap_{i \in I} A_i} \subseteq \bigcap_{i \in I} \overline{A_i}$$
 $x \in$ LHS means than for an arbitrary neighbourhood $N$ of $x$, $N \cap \bigcap_{i \in I} A_i$ is not empty. Therefore $N \cap A_i \forall i \in I$ is non empty, so $x \in \overline{A_i} \forall i \in I$, equivalent to saying $x \in \bigcap_{i \in I} \overline{A_i}$. \\
 LHS does not have to equal RHS, as seen in $\{x \in \R| x < 0\}$ and $\{x \in \R| x > 0\}$ in the space $\R$.
\end{solution}
 \part{How does closure behave with respect to complementation?}
  
\begin{solution}
 \\The complement of the closure is a subset of the closure of the complement, or
 $$X - \overline{A} \subseteq \overline{X - A}$$
 $x \in$ LHS $\Rightarrow x \in X - A \Rightarrow x \in \overline{X - A}$ \\
 LHS does not have to equal RHS, as seen in $A = \{x \in \R| x < 0\}$.
\end{solution}

 \part{How does closure behave with respect to Cartesian products?}
 
 
\begin{solution}
\\Let $A_i$ be a finite collection of sets with metrices $d_i$. Then
$$ \prod_{i \in I} \overline{A_i} = \overline{\prod_{i \in I} A_i}$$
If the metric for the cartesian products is defined as
$$d(\textbf{x}, \textbf{y}) = \sqrt{\sum_{i \in I}d_i^2(x_i, y_i)}$$
First note that "neighbourhoods" can be replaced by "open sets" in \hyperref[thm4.5.8]{Theorem 5.8}. Also, an open hypercube can replace an open sphere in defining open sets under the usual topology. \\
LHS $\subseteq$ RHS: \\
Let $x \in$ LHS, and $C_x$ be an open hypercube centred at $x$ with length $2l$. Since $x \in$ LHS, we know any open set containing $x$ intersects $A_i$, meaning $\exists y_i \in S_l(x_i) \cap A_i$. There exists a point $y$ in the open cube $C_x$ whose $i$th component is in $\overline{A_i}$. Therefore, for an arbitrary $x$ in LHS and an arbitrary open set containing it, the open set also intersects with $\prod_{i \in I} A_i$, meaning $x \in$ RHS. \\
RHS $\subseteq$ LHS: \\
Let $x \in$ RHS, and $S_l(x_i)$ be an arbitrary open interval containing $x_i$. Consider an open hypercube $C_x$ centred at $x$ with length $2l$. Since RHS is a closure, this cube contains $y \in \prod_{i \in I} A_i$, meaning $y_i \in S_l(x_i) \cap A_i$. Since an arbitrary open set containing $x_i$ intersects $A_i$, we have $x_i \in \overline{A_i} \forall i \in I$.
\end{solution}

 \part{How do functions affect closure?} \label{supp4.5.10}
 
 
\begin{solution}
 \\For a continuous function, the function of the closure is a subset of the closure of the function, or
 $$f(\overline{A}) \subseteq \overline{f(A)}$$
 Let $f: X \rightarrow Y$, and $y \in$ LHS. Then $\exists x \in \overline{A}: f(x) = y$. Let $O_y$ be an arbitrary open set in $Y$ containing $y$. Then $O_x = f^{-1}(O_y)$ is an open set containing $x$. Since $x \in$ LHS, $O_x$ must contain an $a \in A$, so $f(a) \in O_y$, meaning $O_y$ intersects $f(A)$. Since this holds $\forall O_y$, $y \in \overline{f(A)}$. \\
 LHS does not have to equal RHS, e.g. when $f(x) = x, X = (\R, \mathbb{U}), Y = (\R, \mathbb{I})$.
\end{solution}
\end{parts}

\subsection{Theorem 5.10}
\setcounter{question}{0}

\question{Let $X$ and $Y$ be topological spaces and let $f: X \rightarrow Y$. Then}

\begin{parts}
 \part{$f$ is continuous if and only if whenever $A \subseteq X$, then $f(\overline{A}) \subseteq \overline{f(A)}$}
 
\begin{solution}
 \\$\Rightarrow$: \\
 See \hyperref[supp4.5.10]{here}. \\
 $\Leftarrow$: \\
 Let $B$ be a closed set in $Y$, and let $A = f^{-1}(B)$. Since $B$ is closed, we have
 $$f(\overline{A}) \subseteq \overline{f(A)} \subseteq B$$
 This means
 $$x \in \overline{A} \Rightarrow f(x) \in B \Rightarrow x \in A$$
 by definition of $A$. Combining this with the fact that $A$ is a subset of $\overline{A}$, we can conclude $A = \overline{A}$, which means $A$ is closed. Under $f$, the inverse of any closed set is still closed, so it is continuous.
\end{solution}

\part{$f$ is continuous if and only if whenever $B \subseteq Y$, then $\overline{f^{-1}(B)} \subseteq f^{-1}(\overline{B})$}

\begin{solution}
 \\$\Rightarrow$: \\
 Let $x \in \overline{f^{-1}(B)}$. Consider an open set $O_y \subseteq Y$ containing $y = f(x)$. Since $f$ is continuous, $O_x = f^{-1}(O_y)$ is open. $O_x$ is an open set that contains $x$, so $\exists b \in O_x:f(b) \in B$. Then $O_y$ intersects $B$. Since $O_y$ is arbitrary, any open set containing $y$ intersects $B$, so $y \in \overline{B}$. Hence $x \in f^{-1}(\overline{B})$. \\
 $\Leftarrow$: \\
 Let $B$ be a closed set in $Y$. Then
 $$\overline{f^{-1}(B)} \subseteq f^{-1}(\overline{B}) = f^{-1}(B)$$
 so
 $$x \in \overline{f^{-1}(B)} \Rightarrow f(x) \in B \Rightarrow x \in f^{-1}(B)$$
Similar to (a), we have shown that the inverse of $B$ is equal to its closure, i.e. it is closed. Since the inverse of any closed set is closed, $f$ is continuous.
\end{solution}
\end{parts}

\subsection{Exercise 5.11}
\setcounter{question}{0}

\question{Show that, fora subset $A$ of a topological space, $A = \overline{A}$ if and only if $A$ is closed.}

\begin{solution}
 \\$\Rightarrow$: \\
 Let $A$ be closed. Then $A$ is the smallest closed set that contains $A$ itself, so it is its own closure. \\
 $\Leftarrow$: \\
 $\overline{A}$ is closed by definition. If $A$ is not closed, $A = \overline{A}$ becomes a contradiction.
\end{solution}

\question{$x \in X$ is a cluster point of $A \subseteq X$ if every neighbourhood of $x$ meets $A$ in at least one point other than $x$ itself. The set of all cluster points of $A$ is called the derived set of $A$ and is denoted $A'$.}

\begin{parts}
 \part{Let $\R$ have its usual topology, $a, b \in \R$ with $a < b$. What is $(a, b)'$? $[a, b)'$? $[a, b]'$? $\{a\}'$?}
 
\begin{solution}
 $[a,b], [a,b], [a,b], \{a\}$
\end{solution}

\part{Show that if $X$ is any topological space, then $\overline{A} = A \cup A'$. Is $A' = \overline{A} - A$?}

\begin{solution}
 Proof for $\overline{A} = A \cup A'$ \\
 LHS $\subseteq$ RHS \\
 Let $x \in$ LHS. Either $x \in A$, or any neighbourhood of $x$ intersects $A$ at a point other than $x$, since $x \notin A$. Hence $x \in A$ or $x \in A'$. \\
 RHS $\subseteq$ LHS \\
 Let $x \in$ RHS. If $x \in A$, the $x \in$ LHS by definition. Or else, any neighbourhood of $x$ intersects with $A$, meaning $x \in \overline{A}$. \\
 $A' = \overline{A} - A$ is false. Consider the usual topology where $A = [0,1]$.
\end{solution}

\part{Let $X = \{a, b, c, d\}$ with the topology $\mathcal{J} = \{\emptyset, X, \{a\}, \{a,b\}, \{c, d\}, \{a, c, d\}\}$. What are $\{p\}'$ and $\overline{\{p\}}$ when $p = a,b,c$, or $d$? What is $\{a,d\}'$? $\overline{\{a,d\}}$? What is $\{b,d\}'$? $\overline{\{b,d\}}$?}

\begin{solution}
 \\$p = a$: $\{b\}, \{a, b\}$ \\
 $p = b$: $\emptyset, \{b\}$ \\
 $p = c$: $\{d\}, \{c,d\}$ \\
 $p = d$: $\{c\}, \{c,d\}$ \\
 $\{a,d\}' = \{b, c\}, \overline{\{a,d\}} = X, \{b,d\}' = \{c\}, \overline{\{b,d\}} = \{b,c,d\}$
\end{solution}
\end{parts}

\question{The largest open set that is contained in a subset $A$ of a topolotical space $X$ is called the interior of $A$, is denoted by $A^\circ$, and is defined by $$A^\circ = \bigcup \{U \subseteq X: U \text{ is open and } U \subseteq A\}$$}

\begin{parts}
 \part{Show that $A^\circ$ is open.}
 
\begin{solution}
 $A^\circ$ is open because it is a union of open sets.
\end{solution}

\part{Show that $A^\circ$ is the largest open set contained in $A$ by showing that $A^\circ \subseteq A$ and if $U \subseteq X$ is open and $U \subseteq A$, then $U \subseteq A^\circ$.}

\begin{solution}
 $A^\circ \subseteq A$ because $x \in A^\circ \Rightarrow x \in U \subseteq A \Rightarrow x \in A$ For the second part of the question, note that the $U$ defined is exactly the $U$ whose union forms $A^\circ$, so $U \subseteq A^\circ$. \\
\end{solution}

\part{Show that $A = A^\circ$ if and only if $A$ is open.}

\begin{solution}
 \\$\Rightarrow$: \\
 Since $A^\circ$ is always open, $A = A^\circ$ implies $A$ is open also. \\
 $\Leftarrow$: \\
 We have proven that $A^\circ \subseteq A$. Since $A$ itself satisfies the definition of $U$, $A$ is part of the union that forms $A^\circ$, hence $A \subseteq A^\circ$. As both sets are subsets or each other, they are equal by definition.
\end{solution}
\end{parts}

\section{Basis for a Topology}

\subsection{Theorem 6.1} \label{thm4.6.1}
\setcounter{question}{0}

\question{A subset of the real line with the usual topology is open if and only if it is the union of a collection of open intervals.}

\begin{solution}
 \\$\Rightarrow$: \\
 Let $A$ be an open subset. For it to be open, $\forall a \in A$ there exists an open interval $O_a$ that is contained in $A$. Hence
 $$A = \bigcup_{a \in A} O_a$$
 which is a union of open intervals. \\
 $\Leftarrow$: \\
 Open intervals are open sets, so a union of them is always open.
\end{solution}

\subsection{Theorem 6.2}
\setcounter{question}{0}

\question{A subset of a metric space $(X,d)$ is open if and only if it is the union of a collection of $r$-balls.}

\begin{solution}
 Replace "interval" with "ball" in \hyperref[thm4.6.1]{Theorem 6.1}.
\end{solution}

\subsection{Exercise 6.4}
\setcounter{question}{0}

\question{Give a basis for $\R$ when it has the}

\begin{parts}
 \part{usual metric topology}
 \part{the indiscrete topology}
 \part{the discrete topology}
 \part{the Sorgenfrey topology}
 \part{the finite-complement topology}
\end{parts}

\begin{solution}
 \\(a): $\mathcal{B} = \{(r-\epsilon, r+\epsilon)|r \in \R, \epsilon \in \R^+\}$ \\
 (b): $\mathcal{B} = \{\R\}$ \\
 (c): $\mathcal{B} = \{\{r\}|r \in \R\}$ \\
 (d): $\mathcal{B} = \{[r, r+\epsilon)|r \in \R, \epsilon \in \R^+\}$ \\
 (e): $\mathcal{B} = \{O| \R-O \text{ is odd}\}$
\end{solution}

\question{}

\begin{parts}
 \part{What is a basis for the order topology on $\Z^+$? On $\R$?}
 
\begin{solution}
 $\{\{n\}|n \in \Z^+\}, \{(r-\epsilon, r+\epsilon)|r \in \R, \epsilon \in \R^+\}$
\end{solution}

\part{What is a basis for the order topology on $[0, \Omega)$?}

\begin{solution}
 $\{(a,b)| a < b \text{ and } a,b \in \Omega\} \cup \{0\}$
\end{solution}

\part{Answer all three questions in (a) and (b) by giving a basis for the order topology on an arbitrary totally ordered set $X$.}

\begin{solution}
\\ If $X$ has no maximum nor minimum, the basis is $\mathcal{B} = \{(a,b)| a < b \text{ and } a,b \in X\}$. If $X$ has a maximum $M$ and/or minimum $m$, the basis is the union between $\mathcal{B}$ and $\{(a, M]| a \neq M \in X\}$ and/or $\{[m, b)| b \neq m \in X\}$.
\end{solution}
\end{parts}

\question{A given topology may have more than one basis.}

\begin{parts}
 \part{Show that the collection of open squares is a basis for the usual metric topology on the plane.}
 
\begin{solution}
 \\In fact, this can be proven for all $\R^n$. Firstly, all open squares are open. For any $x$ in an open square $O_s$, let $r$ be the minimum distance between $x$ and any of the bounds of the square. Then $x \in S_r{x} \subseteq O_s$. Secondly, the union of open squares produce all open sets. Let the centre of the square be $x$, and the side length $l$. Denote the open square by $S'_l(x)$. Given the dimensions, $\exists k \in \R$ such that the distance from $x$ to any corner is $kl$. Then all open spheres contain open squares, namely $S'_{\frac{r}{k}}(x) \subseteq S_r(x)$. Similarly, all open squares contain open spheres, namely $S_{\frac{l}{2}}(x) \subseteq S'_l(x)$. Define functions $f(S_r(x)) = S'_{\frac{r}{k}}(x)$ and $g(S'_l(x)) = S_{\frac{l}{2}}(x)$. Any open set $O = \bigcup_{x \in O} O_x$ defined using open spheres can be rewritten as $\bigcup_{x \in O} f(O_x)$. Similarly, if the same open set is defined using open squares $O_x$, it can be rewritten as $\bigcup_{x \in O} g(O_x)$. Therefore, the topology defined by both open squares and open spheres are the same.
 \end{solution}
 
 \part{Can you think of any other basis for the usual metric topology on the plane?}
 
\begin{solution}
 Open equilateral triangles.
\end{solution}
\end{parts}

\subsection{Theorem 6.5}
\setcounter{question}{0}

\question{Let $(X,d)$ be a metric space. Then the collection $\mathcal{B} = \{S_{\frac{1}{n}}(x):n \in \Z^+ \text{ and } x \in X\}$ is a basis for the metric topology on $(X,d)$.}

\begin{solution}
 \\$\frac{1}{n}$ tends to 0, so $\forall x > 0 \exists n \in \N: x > \frac{1}{n}$. This can be written as a function, where $f(x) = n$ where if $x$ is positive, the inverse of $n$ is smaller than $x$. Now define $g(S_r(x)) = S_{f(r)}(x)$. If $O$ is an open set, it can be written as $\bigcup_{x \in O} O_x$ where $O_x$ are open spheres centred at $x$. The same set can be rewritten as $\bigcup_{x \in O} g(O_x)$. Any union of open sets in $\mathcal{B}$ is obviously open, while any open set has shown to be a union of open sets in $\mathcal{B}$, thus the collection is a basis.
\end{solution}
