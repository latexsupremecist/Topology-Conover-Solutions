\newpage

\section{Topological Spaces and Concepts in General}

\subsection{Exercise 2.6}
\setcounter{question}{0}

\question{Let $X$ be a set. Verify that the indiscrete topology, the discrete topology and the finite-complement topology are in fact topologies on $X$.}

\begin{solution}
 \\Indiscrete Topology: \\
 $\emptyset$ and $X$ are open sets, and any intersection/union of open sets are obviously either empty of $X$. \\
 Discrete Topology: \\
 $\emptyset$ and $X$ are open sets. Since any subset is open, any intersection/union of open sets must be a subset, and hence is open. \\
 Finite-complement Topology: \\
 $\empty set$ and $X$ are open sets. Let $A_i$ denote open sets. Then $X - \bigcup_i A_i \subseteq X - A_i$, where $X - A_i$ has a finite cardinality by definition. Therefore, $X - \bigcup_i A_i$ also has a finite cardinality, so $\bigcup_i A_i$ is open. Now let $B_i$ denote finitely many open sets. $X-B_i$ is finite, and so is $\bigcup_i (X - B_i)$ (finite union of finite sets is finite). Since $\bigcup_i (X - B_i) = X - \bigcap_i B_i$ which is finite, $\bigcap_i B_i$ is an open set.
\end{solution}

\question

\begin{parts}
 
 \part{Verify that Sierpinski space is a topological space.}
 
 
\begin{solution}
It contains $\emptyset$ and $X$. Since there are only 3 open sets, brute forcing through all possible unions/intersections show that they are also open sets.
\end{solution}

\part{We said that there are only three different topologies that can be assigned to the 2 point set $\{0,1\}$.  Is the collection of $\{\emptyset, \{1\},\{0,1\}\}$ one of those three topologies on $\{0,1\}$?}

\begin{solution}
 Yes. The same approach for (a) can be used here, since this is essentially the Sierpinski space with 0 and 1 reversed.
\end{solution}

\part{What is $\{0,1\}$ with the finite-complement topology?}

\begin{solution}
 $\{\emptyset, \{0\}, \{1,\}, \{0,1\}\}$
\end{solution}
\end{parts}

\question{List all topologies that can be assigned to a 3 point set.}

\begin{solution}
 \\$\{\emptyset, \{0,1,2\}\}$ \\
 $\{\emptyset, \{0\}, \{0,1,2\}\}$ \\
 $\{\emptyset, \{1\}, \{0,1,2\}\}$ \\
 $\{\emptyset, \{2\}, \{0,1,2\}\}$ \\
 $\{\emptyset, \{0,1\}, \{0,1,2\}\}$ \\
 $\{\emptyset, \{0,2\}, \{0,1,2\}\}$ \\
 $\{\emptyset, \{1,2\}, \{0,1,2\}\}$ \\
 $\{\emptyset, \{0\}, \{0,1\}, \{0,1,2\}\}$ \\
 $\{\emptyset, \{0\}, \{0,2\}, \{0,1,2\}\}$ \\
 $\{\emptyset, \{1\}, \{0,1\}, \{0,1,2\}\}$ \\
 $\{\emptyset, \{1\}, \{1,2\}, \{0,1,2\}\}$ \\
 $\{\emptyset, \{2\}, \{0,2\}, \{0,1,2\}\}$ \\
 $\{\emptyset, \{2\}, \{1,2\}, \{0,1,2\}\}$ \\
 $\{\emptyset, \{0\}, \{1\}, \{0,1\}, \{0,1,2\}\}$ \\
 $\{\emptyset, \{0\}, \{2\}, \{0,2\}, \{0,1,2\}\}$ \\
 $\{\emptyset, \{1\}, \{2\}, \{1,2\}, \{0,1,2\}\}$ \\
 $\{\emptyset, \{0\}, \{0,1\}, \{0,2\}, \{0,1,2\}\}$ \\
 $\{\emptyset, \{1\}, \{0,1\}, \{1,2\}, \{0,1,2\}\}$ \\
 $\{\emptyset, \{2\}, \{0,2\}, \{1,2\}, \{0,1,2\}\}$ \\
 $\{\emptyset, \{0\}, \{1\}, \{0,1\}, \{0,2\}, \{0,1,2\}\}$ \\
 $\{\emptyset, \{0\}, \{2\}, \{0,1\}, \{0,2\}, \{0,1,2\}\}$ \\
 $\{\emptyset, \{0\}, \{1\}, \{0,1\}, \{1,2\}, \{0,1,2\}\}$ \\
 $\{\emptyset, \{1\}, \{2\}, \{0,1\}, \{1,2\}, \{0,1,2\}\}$ \\
 $\{\emptyset, \{0\}, \{2\}, \{0,2\}, \{1,2\}, \{0,1,2\}\}$ \\
 $\{\emptyset, \{1\}, \{2\}, \{0,2\}, \{1,2\}, \{0,1,2\}\}$ \\
 $\{\emptyset, \{0\}, \{1\}, \{2\}, \{0,1\}, \{0,2\}, \{1,2\}, \{0,1,2\}\}$ \\
\end{solution}

\question{Verify that the Sorgenfrey topology defined on the real line is in fact a topology. Is the interval $(0,1)$ open in this topology? How about $(0,1]$? Is $[0,1]$ closed?}

\begin{solution}
 \\The Sorgenfrey topology obviously contains $\emptyset$ and $\R$. Let $A_i$ denote open sets. Then
 $$\forall x \in \bigcup_i A_i, x \in A_i \Rightarrow x \in [a,b) \subseteq A_i \subseteq \bigcup_i A_i$$
 so any union of open sets is open. Now let $B_i$ denote finitely many open sets. Now if $x \in B_i$, we let $\{p_i,q_i\} \subset \R$ such that $x \in [p_i,q_i) \subseteq B_i$. Let $P = \{p_i\}$ and $Q = \{q_i\}$. Since both $P$ and $Q$ are finite, $P$ has a maximum $p$ and $Q$ has a minimum $Q$.
 Now
 $$x \in \bigcap_i B_i \Rightarrow x \in [p,q) \subseteq [p_i,q_i) \subseteq B_i \forall i$$
 Since $[p,q)$ is a subset of $B_i \forall i$, it is a subset of $\bigcap_i B_i$, hence any finite intersection of open sets is open. This shows that the Sorgenfrey topology is a topology. \\
 $(0,1)$ is open, because
 $$\forall x \in (0,1), x \in [x,1) \subset (0,1)$$
 $(0,1]$ is not open, because $1 \in (0,1]$, but if $1 \in [a,b)$, then $\frac{1+b}{2}$ is an element of $[a,b)$ but not $(0,1]$, so $1 \in [a,b) \subseteq (0,1]$ cannot be true. \\
 Consider $A = \R - [0,1]$ and let $x \in A$. Either $x < 0$, and $x \in [x,\frac{x}{2}) \subset A$, or $x > 1$, and $x \in [x,x+1)$. Therefore, $A$ is open, so $[0,1]$ is closed.
\end{solution}

\question{Consider the topological spaces $(\R, \mathcal{I}), (\R, \mathcal{D}), (\R, \mathcal{U}), (\R, \mathcal{S})$ and with the finite-complement topology (where $\mathcal{U}$ denotes the usual topology on $\R$ s in Chapter 3).}

\begin{parts}
 
 \part{If $p \in \R$, is $\{p\}$ open in any of these spaces? Which ones?}
 
 
\begin{solution}
 $(\R, \mathcal{D})$. 
\end{solution}

\part{If $p \in \R$, is $\{p\}$ closed in any of these spaces? Which ones?}

\begin{solution}
$(\R, \mathcal{D}), (\R, \mathcal{U}), (\R, \mathcal{S})$, and the finite-complement topology.
\end{solution}

\part{In which of these spaces is $(a,b)$ open? $[a,b)$? $(a,b]$? $[a,b]$?}

\begin{solution}
 \\$(a,b)$: \\
 $(\R, \mathcal{D}), (\R, \mathcal{U}), (\R, \mathcal{S})$ \\
 $[a,b)$: \\
 $(\R, \mathcal{D}), (\R, \mathcal{S})$ \\
 $(a,b]$: \\
 $(\R, \mathcal{D})$ \\
 $[a,b]$: \\
 $(\R, \mathcal{D})$ \\
\end{solution}

\part{Is the set $\{x \in \R: x \neq \frac{1}{n}\}$ open in any of the spaces? Is it closed in any of them?}

\begin{solution}
 \\Open: \\
 $(\R, \mathcal{D})$ \\
 Closed: \\
 $(\R, \mathcal{D})$ \\
\end{solution}

\part{Is the set $\{x \in \R: x \neq \frac{1}{n} \text{ and } x \neq 0\}$ open in any of the spaces? Is it closed in any of them?}

\begin{solution}
 \\Open: \\
 $(\R, \mathcal{D})$ \\
 Closed: \\
 $(\R, \mathcal{D}), (\R, \mathcal{U}), (\R, \mathcal{S})$ \\
\end{solution}

\end{parts}

\question{Consider the spaces of Problem 5 above again, together with the three spaces that can be defined on $\{0,1\}$.}

\begin{parts}
 
 \part{In which of these spaces are true: If $x$ and $y$ are two distinct points in the space then either there exists an open set $U$ such that $x \in U$ and $y \notin U$, or there exists an open set $V$ such that $y \in V$ and $x \notin V$. (A space for which this statement holds is called a T$_0$-space.)}
 
\begin{solution}
 $(\R, \mathcal{D}), (\R, \mathcal{U}), (\R, \mathcal{S})$, the finite-complement topology, and the Sierpinski space.
\end{solution}

\part{In which of these spaces is the following statement true: If $x$ and $Y$ are two distinct points in the space, then there exists an open set $U$ such that $x \in U$ and $y \notin U$, and there exists an open set $v$ such that $y \in V$ and $x \notin V$. (A space for which this statement holds is called a T$_1$-space.)}

\begin{solution}
 $(\R, \mathcal{D}), (\R, \mathcal{U}), (\R, \mathcal{S})$, and the finite-complement topology.
\end{solution}

\part{In which of these spaces is the following statement true: If $x$ and $y$ are two distinct points in the space, the there exist open sets $U$ and $V$ such that $x \in U$, $y \in V$ and $U \cap V = \emptyset$. (A space for which this statement holds is called a T$_2$-space or a Hausdorff space.)}

\begin{solution}
  $(\R, \mathcal{D}), (\R, \mathcal{U}), (\R, \mathcal{S})$
\end{solution}

\end{parts}

\question{Show that every T$_2$-space is a T$_1$-space, and that every T$_1$-space is a T$_0$-space, and give an example of a T$_0$-space that is not a T$_1$-space, and an example of a T$_1$-space that is not a T$_2$-space.}

\begin{solution}
 \\T$_2$-space $\Rightarrow$ T$_1$-space: Since $U \cap V \neq \emptyset$, $x \notin V$ and $y \notin U$, so $V$ and $U$ are open sets that make the space a T$_1$-space. \\
T$_1$-space $\Rightarrow$ T$_0$-space: either one of $U$ and $V$ make the space a T$_0$-space. \\
T$_0$-space that is not a T$_1$-space: the Sierpinski space \\
T$_1$-space that is not a T$_2$-space: the finite-complement topology \\
\end{solution}

\question{A topological space $X$ is said to be \textbf{metrizable} if a metric can be defined on $X$ so that a set is open in the metric topology induced by this metric if and only if it is open in the topology that is already on the space.}

\begin{parts}
 
 \part{Let $X$ be a set with more than one point. Prove that $(X, \mathcal{I})$ is \textit{not} metrizable. Thus the indiscrete topology on a set with more than one point is an example of a topological space that is \textit{not} a metric space.}
 
 
\begin{solution}
 \\If $X$ has more than one point, it has 2 distinct points $x$ and $y$, where we let $r = d(x,y) > 0$ by the definition of a metric. $S_{\frac{r}{2}}(x)$ is an open space according to the metric. However, it is neither empty (contains $x$) nor the universe (does not contain $y$). This forms a contradiction.
\end{solution}

\part{Let $X$ be a set. Define a function from $X \times X = \{(x,y):x,y\in X\}$ to $\R$ by $$d(x,y) = \begin{cases} 1 & \text{if } x \neq y \\ 0 \text{if } x = y \end{cases}$$ Prove that $d$ is a metric on $X$. What is the metric topology induced by $d$?}

\begin{solution}
 \\Obviously, $d(x,y) = 0$ if and only if $x=y$, $d(x,y) \geq 0$, $d(x,y) = d(y,x)$. For the triangle inequality $d(x,y) \leq d(x,z) + d(y,z)$, note that it is trivial if $x=y$. If not, at least one of $d(x,z)$ and $d(y,z)$ must be nonzero, so the inequality holds. Therefore, $d(x,y)$ is a metric. \\
 $\forall x \in X, S_{0.5}(x) = \{x\}$. Since all singleton sets are open, the metric topology induced by $d$ is the discrete topology.
\end{solution}
\end{parts}

\subsection{Theorem 3.2} \label{thm4.3.2}
\setcounter{question}{0}

\question{Let $(X, d_1)$ and $(Y, d_2)$ be metric spaces and let $f: X \rightarrow Y$. Then $f$ is continuous at $x_0 \in X$ if and only if whenever $V$ is an open subset of $Y$ with $f(x_0) \in V$, then there exists an open subset $U$ of $X$ such that $x_0 \in U$ and $f(U) \subseteq V$.}
 
 
\begin{solution}
 \\$\Rightarrow$: \\
 Let $V$ be open. Since $V$ is open, $\forall v = f(x_0) \in V$, $\exists \epsilon > 0$ where $S_\epsilon(v) \subseteq V$. Since continuity is implied, $\exists \delta > 0$ where $f(S_\delta(x_0)) \subseteq S_\epsilon(v)$. Therefore $S_\delta(x_0)$ is the desired open $U$. \\
 $\Leftarrow$: \\
 Let $f(x_0) = v$. $\forall \epsilon > 0$, $V = S_\epsilon(v)$ is open. Then an open $U$ exists where $x_0 \in U$. By definition, U is open, so $\exists \delta > 0$ such that $S_\delta(x_0) \in U$. Then
 $$f(S_\delta(x_0)) \subseteq f(U) \subseteq V = S_\epsilon(v)$$
 This demonstrates that $f$ is continuous by the $\epsilon-\delta$ definition.
\end{solution}

\subsection{Theorem 3.4}
\setcounter{question}{0}

\question{Let $X$ and $Y$ be topological spaces and let $f:X \rightarrow Y$. Then $f$ is continuous on $X$ if and only if whenever $V$ is an open subset of $Y$, then $f^{-1}(V)$ is open in $X$.}

\begin{solution}
 \\$\Rightarrow$: \\
 Based on \hyperref[thm4.3.2]{Theorem 3.2}, we have an open $U_x \forall f(x) \in V$ such that $f(U_x) \subseteq V$. Therefore, $U_x \subseteq f^{-1}(V)$. Then
 $$f^{-1}(V) = \bigcup_{x \in f^{-1}(V)} U_x$$
 Since $f^{-1}(V)$ is a union of open sets, it is open. \\
 $\Leftarrow$: \\
 Similar to the second part of \hyperref[thm4.3.2]{Theorem 3.2}, let $f(x_0) = v$. $\forall \epsilon > 0$, $V = S_\epsilon(v)$ is open. Then $f^{-1}(V)$ is open (and contains $x_0$), so $\exists \delta > 0$ such that $S_\delta(x_0) \in f^{-1}(V)$. Then
 $$f(S_\delta(x_0)) \subseteq V = S_\epsilon(v)$$
 This demonstrates that $f$ is continuous by the $\epsilon-\delta$ definition.
\end{solution}

\subsection{Exercise 3.5}
\setcounter{question}{0}

\question{Consider $(\R, \mathcal{I}), (\R, \mathcal{D}), (\R, \mathcal{U}), (\R, \mathcal{S})$ and $(\R, \mathcal{F})$, the real line with the indiscrete topology, the discrete topology, the usual metric topology, the Sorgenfrey topology and the finite-complement topology, respectively. Let $f: \R \rightarrow \R$ be the identity function defined by $f(r)=r$ for all real numbers $r$. Determine all possible choices for $\mathcal{J}_1$ and $\mathcal{J}_2$ from $\mathcal{I}, \mathcal{D}, \mathcal{U}, \mathcal{S}, \mathcal{F}$ so that $f:(\R, \mathcal{J}_1) \rightarrow (\R, \mathcal{J}_2)$ is continuous.}

\begin{solution}
 
 \begin{center}
\begin{tabular}{|c||c|c|c|c|c|}
\hline
 & $\mathcal{I}$ & $\mathcal{D}$ & $\mathcal{U}$ & $\mathcal{S}$ & $\mathcal{F}$ \\
 \hline \hline
 $\mathcal{I}$ & \cmark & \xmark & \xmark & \xmark & \xmark \\
 \hline
 $\mathcal{D}$ & \cmark & \cmark & \cmark & \cmark & \cmark \\
 \hline
 $\mathcal{U}$ & \cmark & \xmark & \cmark & \xmark & \cmark \\
 \hline
 $\mathcal{S}$ & \cmark & \xmark & \cmark & \cmark & \cmark \\
 \hline
 $\mathcal{F}$ & \cmark & \xmark & \xmark & \xmark & \cmark \\
 \hline
\end{tabular}
\end{center}
 Where the rows denote $\mathcal{J}_1$ and the columns denote $\mathcal{J}_2$. 
\end{solution}

\question{Let $x$ be a set and let $\mathcal{J}$ be any topology on $X$. There is a topology $\mathcal{J}$' that can be assigned to $X$ so that the identity function from $(X, \mathcal{J}')$ to $(X\mathcal{J})$ is always continuous no matter what $\mathcal{J}$ s. What is it?}

\begin{solution}
 The discrete topology $\mathcal{D}$.
\end{solution}

\question{Let $X$ be any set and let $\mathcal{J}$ be any topology on $X$. There is a topology $\mathcal{J}'$ that can be given to $X$ so that the identity function from $(X, \mathcal{J})$ to $(X, \mathcal{J}')$ is always continuous no matter what $\mathcal{J}$ is. What is it?}

\begin{solution}
 The indiscrete topology $\mathcal{I}$.
\end{solution}

\question{A function that preserves open sets is called an \textbf{open function}. More precisely, a function $f:X \rightarrow Y$ is an open function if whenever $U$ is open in $X$, then $f(U)$ is open in $Y$.}

\begin{parts}
 
 \part{Give an example of a continuous function that is not open.}
 
\begin{solution}
 $f(r) = r$ where $X$ is the real line with the discrete topology and $Y$ is the real line with the indiscrete topology.
\end{solution}

\part{Give an example of an open function that is not continuous.}

\begin{solution}
 $f(r) = r$ where $X$ is the real line with the indiscrete topology and $Y$ is the real line with the discrete topology.
\end{solution}

\end{parts}